\noindent The surge forces from the propeller are taken as the propeller thrust multiplied by a thrust deduction factor $t_{df}$ as,
\begin{equation}
    \label{eq:X_P}
    X_P = (1-t_{df})T
\end{equation}
where in this paper the propeller thrust $T$ is proposed to be taken from the measured thrust from either virtual captive tests (VCTs) or free-running model tests (FRMTs) to reduce the uncertainty of the complex interaction between the propeller, rudder, and hull. The propeller also generates side forces, especially for yaw rates, which have a small stabilizing effect on the ship. The stabilizing propeller moment can be around 5\% of the rudder yawing moment, as shown in \autoref{fig:propeller_size_force}.
This effect is not explicitly modeled in this paper, so that $Y_P=0$,$N_P=0$. The propeller side force is however included in the VCT data, so that propeller side force will be included in the hull coefficients instead.

\begin{figure}[h!]
    \centering   
    \includesvg[width=4in]{figures/model_propeller_side_force.propeller_size_force.svg}
    \caption{Typical yawing moments from rudder and propeller for various yaw rates.}
    \label{fig:propeller_size_force}
\end{figure}