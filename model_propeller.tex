The surge forces from the propeller are taken as the propeller thrust multiplied by a thrust deduction factor $t_{df}$ (\autoref{eq:X_P}).
\begin{equation}
    \label{eq:X_P}
    X_P = (1-t_{df})T
\end{equation}
The propeller thrust $T$ is taken as the measured thrust from VCT or FRMTs in this paper, to reduce the uncertainty of the complex interaction between the propeller, rudder, and hull. The propeller also generates side forces, especially in oblique or circular inflows, which have a small stabilizing effect on the ship. 
\begin{figure}[h]
    \centering
    \includesvg{figures/model_propeller_side_force.propeller_size_force.svg}
    \caption{Typical yawing moments from rudder and propeller.}
    \label{fig:propeller_size_force}
\end{figure}
The sway force and yawing moment are both set to zero $Y_P=0$,$N_P=0$. 