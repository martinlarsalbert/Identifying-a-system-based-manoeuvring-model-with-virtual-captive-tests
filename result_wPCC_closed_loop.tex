\noindent As a complement to the inverse dynamics analysis, closed loop simulations were conducted for the wPCC. The simulation results are shown in \autoref{fig:sim_wPCC}. As expected from the inverse dynamic analysis presented above, differences could also be observed in the simulation comparison FRMTs. 
The zigzag periods of the semi-empirical model were too short compared to the experiments and the overshoot angles were over-predicted up to three degrees. For the 10/10 zigzag test, by using the semi-empirical rudder to perform the close-loop maneuvering simulations, the over-prediction for both the first and second overshoot angles is about 1 degree. While for the 20/20 zigzag test, there is an increasing trend of over-prediction angles, i.e., less than 2 degrees over-prediction for the first overshoot angle, but up to 3 degrees discrepancy for the second overshoot angle. As explained before, this might be caused by the semi-empirical models to estimate rudder forces, as well as the asymmetric flow applied on the two rudders during the drifting operations.
\begin{figure}[h]
     \centering
     \begin{subfigure}[b]{\textwidth}
         \centering
         \includesvg[height=0.3\textheight]{figures/results_wPCC_ID.closed loop zigzag 10_10 port.svg}
        \caption{Simulations wPCC Zigzag10/10 to port.}
        \label{fig:sim_wPCC_10}
     \end{subfigure}
     \vfill
     \begin{subfigure}[b]{\textwidth}
        \centering
        \includesvg[height=0.3\textheight]{figures/results_wPCC_ID.closed loop zigzag 20_20 stbd.svg}
        \caption{Simulations wPCC Zigzag20/20 to starboard.}
        \label{fig:sim_wPCC_20}
     \end{subfigure}
        \caption{Comparison between zigzag tests with wPCC from experiments and simulations with a model equipped with either a polynomial rudder or semi-empirical rudder model.}
        \label{fig:sim_wPCC}
\end{figure}
%\begin{figure}[h]
%     \centering
%     \begin{subfigure}[b]{\textwidth}
%         \centering
%         \includesvg{figures/results_wPCC_ID.overshoot1.svg}
%        \caption{First overshoot angles.}
%        \label{fig:overhoots1_wPCC}
%     \end{subfigure}
%     \vfill
%     \begin{subfigure}[b]{\textwidth}
%         \centering
%         \includesvg{figures/results_wPCC_ID.overshoot2.svg}
%        \caption{Second overshoot angles.}
%        \label{fig:overhoots2_wPCC}
%     \end{subfigure}
%     
%        \caption{Overshoot angles from the wPCC experiments and simulations.}
%        \label{fig:overshoots_wPCC}
%\end{figure}