As a complement to the inverse dynamics analysis, closed loop simulations were conducted for the wPCC. The simulation results are shown in \autoref{fig:sim_wPCC}. The polynomial rudder has the best agreement with the experiments. The zigzag period of the semi-empirical model was too short compared to the experiments as a result of the over predicted rudder yawing moment $Y_R$, that was revealed in the inverse dynamics analysis. 

The predicted overshoot angles are however quite similar to the experiments for both the models, as shown in \autoref{fig:overshoots_wPCC}.
\begin{figure}[h]
     \centering
     \begin{subfigure}[b]{\textwidth}
         \centering
         \includesvg[height=0.3\textheight]{figures/results_wPCC_ID.closed loop zigzag 10_10 port.svg}
        \caption{Simulations wPCC Zigzag10/10 to port.}
        \label{fig:sim_wPCC_10}
     \end{subfigure}
     \vfill
     \begin{subfigure}[b]{\textwidth}
        \centering
        \includesvg[height=0.3\textheight]{figures/results_wPCC_ID.closed loop zigzag 20_20 stbd.svg}
        \caption{Simulations wPCC Zigzag20/20 to starboard.}
        \label{fig:sim_wPCC_20}
     \end{subfigure}
        \caption{Comparison between zigzag tests with wPCC from experiments and simulations with a model equipped with either a polynomial rudder or semi-empirical rudder model.}
        \label{fig:sim_wPCC}
\end{figure}
\begin{figure}[h]
     \centering
     \begin{subfigure}[b]{\textwidth}
         \centering
         \includesvg{figures/results_wPCC_ID.overshoot1.svg}
        \caption{First overshoot angles.}
        \label{fig:overhoots1_wPCC}
     \end{subfigure}
     \vfill
     \begin{subfigure}[b]{\textwidth}
         \centering
         \includesvg{figures/results_wPCC_ID.overshoot2.svg}
        \caption{Second overshoot angles.}
        \label{fig:overhoots2_wPCC}
     \end{subfigure}
     
        \caption{Overshoot angles from the wPCC experiments and simulations.}
        \label{fig:overshoots_wPCC}
\end{figure}