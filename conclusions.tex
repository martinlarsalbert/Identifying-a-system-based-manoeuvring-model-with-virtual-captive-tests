%_________________________________________________________
%Move 1: Background information (research purposes, theory,
%methodology)
%
\noindent Maneuvering models were developed for two WAPS test cases with large rudders. The models were identified by conducting VCTs to get the hydrodynamic damping coefficients and by conducting FNPF pure yaw and pure sway tests to get the added masses. The identified force models were compared with inverse dynamics forces from zigzag tests, to identify possible weak spots within the models.  
%%_________________________________________________________
%%Move 2: Summarizing and reporting key results. (oblig.)
The added masses from the FNPF pure yaw and pure sway tests produced inverse dynamics forces that agreed well with the force prediction models and the state VCTs for Optiwise, which suggests that this is an accurate way to determine the added masses.
Two enhancements to the MMG rudder model were proposed which improved the fit to the VCT data and also gave a little bit better results in the closed loop simulations.
The coupling terms between sway and yaw rate was found to be important for both wPCC and Optiwise.

The method to use inverse dynamics in combination with the state VCTs was found to be a useful tool to test the identified models and see if errors origin from the model structure or were inherent in the VCT data.
It was for instance shown that the state VCT did not agree well with the wPCC inverse dynamics forces, which indicates that there is something missing in the wPCC VCT data. This could for instance be the lack of wave generation or heeled cases in the VCT data, which might be too much of a simplification for the wPCC.
For Optiwise on the other hand, which was run at much lower Froude number and little heel, the state VCT agreed much better with the inverse dynamics forces and better agreement was obtained in the closed loop simulations.
%_________________________________________________________
%Move 3: Commenting on key results (making claims, explaining the results,
%comparing the new work with previous studies, offering
%alternative explanations) (oblig.)

%_________________________________________________________
%Move 4: Stating the limitations of the study
The simulation model in this paper were not equipped with a propeller prediction model, instead measured thrust was used during the simulations. This needs to be added to the models in order to have self containing models that can run new simulations. 
%_________________________________________________________
%Move 5: Making recommendations for future implementation and/or for
%future research
