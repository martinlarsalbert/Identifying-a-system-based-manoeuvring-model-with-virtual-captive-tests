%_________________________________________________________
%Move 1: Background information (research purposes, theory,
%methodology)
%
\noindent Maneuvering models were developed for two WAPS test cases with large rudders. The models were identified by conducting VCTs to obtain the hydrodynamic damping coefficients and by conducting pure yaw and pure sway simulations in the FNPF to obtain the added masses. The identified force models were compared with the inverse dynamics forces of the zigzag tests to identify possible weaknesses within the models.  

%%_________________________________________________________
%%Move 2: Summarizing and reporting key results. (oblig.)
The identified wPCC model was shown to be well adapted to the VCT data, where the importance of the coupling terms was also shown in the combined drift and yaw rate cases.
However, the FRMT inverse dynamics forces were quite different from the VCT data and the model predictions. From this it was concluded that there must be an inherent error in the VCT data for wPCC that was inherited by the identified hull force model.
The inherent error in the wPCC VCT data could originate from false assumptions about neglected wave generation and roll exclusion. Therefore, another test case, Optiwise, was investigated in which these assumptions were assumed to be more valid.

It was shown that a model, with the rudder model replaced by the actual measured rudder forces, predicted total forces that were similar to the total forces from the inverse dynamics. Hence, the Optiwise hull force prediction was considered valid.
The newly proposed MMG quadratic rudder model was then shown to be identifiable from the VCT data to get good agreement with the FRMT measured rudder forces and consequently also good agreement with the total forces. 
%_________________________________________________________
%Move 3: Commenting on key results (making claims, explaining the results,
%comparing the new work with previous studies, offering
%alternative explanations) (oblig.)
For Optiwise it can therefore be concluded that the VCT data contained correct damping forces during the maneuvers, which were well described by the chosen model structure, and also that the method used to determine added masses gave reasonable values. 

This analysis of the wPCC and Optiwise test cases has shown that inverse dynamics together with state VCTs is an effective tool to identify and explain possible weaknesses within the identified models.
%It was shown that the proposed way to determine added masses with FNPF pure yaw and pure sway tests
%
%added masses from the FNPF pure yaw and pure sway tests produced inverse dynamics forces that agreed well with the force prediction models and the state VCTs for the Optiwise test case, which suggests that this is an accurate way to determine the added masses.
%
%Two enhancements to the MMG rudder model were proposed which improved the fit to the VCT data and also gave a little bit better results in the closed loop simulations.
%The coupling terms between sway and yaw rate were important for both wPCC and Optiwise.

%_________________________________________________________
%Move 4: Stating the limitations of the study
It should also be mentioned that if completely new maneuvers are simulated, the models in this paper would need a propeller prediction model that would be self-containing.
%_________________________________________________________
%Move 5: Making recommendations for future implementation and/or for
%future research
