Results from closed loop simulations with Optiwise are shown in \autoref{fig:sim_optiwise}. The polynomial rudder model has longer zigzag periods compared to the experiments and the other models, especially during the second turns. This can be explained by the under predicted yawing moment. The predicted overshoot angles are shown in \autoref{fig:overshoots_optiwise}.   
\begin{figure}[h]
     \centering
     \begin{subfigure}[b]{0.40\textwidth}
         \centering
         \includesvg{figures/results_optiwise_ID.closed loop zigzag 10_10 port.svg}
        \caption{Zigzag10/10 to port.}
        \label{fig:sim_optiwise_10_port}
     \end{subfigure}
     \hfill
     \begin{subfigure}[b]{0.40\textwidth}
         \includesvg{figures/results_optiwise_ID.closed loop zigzag 10_10 stbd.svg}
        \caption{Zigzag10/10 to starboard.}
        \label{fig:sim_optiwise_10_stbd}
     \end{subfigure}
     \vfill
     \begin{subfigure}[b]{0.40\textwidth}
         \centering
         \includesvg{figures/results_optiwise_ID.closed loop zigzag 20_20 port.svg}
        \caption{Zigzag20/20 to port.}
        \label{fig:sim_optiwise_20_port}
     \end{subfigure}
     \hfill
     \begin{subfigure}[b]{0.40\textwidth}
         \includesvg{figures/results_optiwise_ID.closed loop zigzag 20_20 stbd.svg}
        \caption{Zigzag20/20 to starboard.}
        \label{fig:sim_optiwise_20_stbd}
     \end{subfigure}
     
        \caption{Comparison between zigzag tests with Optiwise from experiments and simulations with a model equipped with the MMG rudder models.}
        \label{fig:sim_optiwise}
\end{figure}
\begin{figure}[h]
     \centering
     \begin{subfigure}[b]{\textwidth}
         \centering
         \includesvg{figures/results_optiwise_ID.overshoot1.svg}
        \caption{First overshoot angles.}
        \label{fig:overhoots1_optiwise}
     \end{subfigure}
     \vfill
     \begin{subfigure}[b]{\textwidth}
         \centering
         \includesvg{figures/results_optiwise_ID.overshoot2.svg}
        \caption{Second overshoot angles.}
        \label{fig:overhoots2_optiwise}
     \end{subfigure}
     
        \caption{Overshoot angles from the Optiwise experiments and simulations.}
        \label{fig:overshoots_optiwise}
\end{figure}