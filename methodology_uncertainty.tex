\noindent The repeatability of the experiments was assessed by three repeated 20/20 zigzags to starboard (A--C) for the Optiwise test case, as shown in \autoref{fig:repeatability}. A minor time difference was observed for the first counter-rudder action around t=9 s, likely due to slight variations in the initial conditions of the tests. \autoref{fig:diff} presents the differences between repeated tests, expressed as deviations from the first test (A). The deviations for the total sway force $Y_D$ and the total yawing moment $N_D$ follow an approximately normal distribution, with standard deviations of 1 N and 1.3 Nm, respectively. These values correspond to 1.7\% and 1.2\% of the maximum values in $Y_D$ and $N_D$. A 5\% deviation in $Y_D$ and a 3.5\% deviation in $N_D$, representing approximately three standard deviations, are used to express the uncertainty of the experiments (UE).
\begin{figure}[h!]
    \centering   
    \includesvg{figures/methodology_repeatability.zigzag.svg}
    \caption{Zigzag 20/20 repeatability tests.}
    \label{fig:repeatability}
\end{figure}
\begin{figure}[h!]
    \centering   
    \includesvg{figures/methodology_repeatability.diff_all.svg}
    \caption{Differences between repeatability tests.}
    \label{fig:diff}
\end{figure}
%\begin{figure}[h!]
%    \centering   
%    \includesvg{figures/methodology_repeatability.diff1.svg}
%    \caption{.}
%    \label{fig:diff1}
%\end{figure}
%
%\begin{figure}[h!]
%    \centering   
%    \includesvg{figures/methodology_repeatability.diff2.svg}
%    \caption{.}
%    \label{fig:diff2}
%\end{figure}