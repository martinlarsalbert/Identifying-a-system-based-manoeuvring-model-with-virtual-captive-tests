\noindent Some variables in the equations are expressed as non-dimensional units using the prime system, denoted by the prime symbol ($'$). The variables are converted from SI units to the prime system using the denominators in \autoref{tab:prime_system} for the corresponding physical quantity, where $U$ and $L$ are the velocity and length between the perpendiculars of the ship, respectively, and $\rho$ is the water density.
For the calculation of surge velocity $u'$, the perturbed velocity $(u-U_0)$ about a nominal speed $U_0$ is used, as in \autoref{eq:u_prime}, to avoid a $u'$ of 1 for all speeds when the ship is on a straight course (where $u=U$), as in a resistance or self-propulsion test, 
\begin{equation}
    \label{eq:u_prime}
    u' = \frac{u-U_0}{U}, \hspace{0.1cm}
    F_{n0} = \frac{U_0}{\sqrt{g \cdot L}}
\end{equation}
where $U_0$ is instead expressed as a Froude number for a non-dimensional variable.
% \begin{equation}
%     \label{eq:Fn0}
%     F_{n0} = \frac{U_0}{\sqrt{g \cdot L}}
% \end{equation}
The usage of the perturbed velocity, therefore, allows for higher-order resistance terms in the model, such as $X_{u}$, which are otherwise not possible. 