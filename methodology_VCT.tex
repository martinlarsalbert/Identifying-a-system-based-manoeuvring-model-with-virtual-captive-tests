\noindent In this study, it is proposed to conduct VCTs by solving a set of static flow calculations with CFD to calculate the hydrodynamic forces applied on the ship hull. To have good coverage of the states that the ship will have during the maneuvering, the VCT test matrices listed in (\autoref{tab:VCT_wPCC}, \autoref{tab:VCT_optiwise}) are proposed in this study. In addition, for the wPCC and Optiwise specially designed with WAPS, the combinations of drift angle and yaw rate are also proposed as in \autoref{fig:phase_plots}. 


From the series of the VCTs, the hull forces  $X_{VCT}$, $Y_{VCT}$, and $N_{VCT}$ can be obtained. They are then used to recalculate the damping forces as follows,
\begin{equation}
    \label{eq:X_D}
    \left.\begin{aligned}
    X_{D} = X_{VCT} + Y_{\dot{r}} r^{2} + Y_{\dot{v}} r v \\
    Y_{D} =  Y_{VCT} - X_{\dot{u}} r u\\
    N_{D} = N_{VCT} + X_{\dot{u}} u v - Y_{\dot{r}} r u - Y_{\dot{v}} u v
    \end{aligned}\right\}
\end{equation}
% \begin{equation}
%     \label{eq:Y_D}
%     Y_{D} = - X_{\dot{u}} r u + Y_{VCT}
% \end{equation}
% \begin{equation}
%     \label{eq:N_D}
%     N_{D} = N_{VCT} + X_{\dot{u}} u v - Y_{\dot{r}} r u - Y_{\dot{v}} u v
% \end{equation}

It should be noted that the mass $m$ has disappeared from Eq.(\ref{eq:F_expanded}) to derive these expressions, because the ship is not moving in the VCTs conducted by the Shipflow Motions, instead the water has either oblique or circular inflow \citep{roychoudhuryCFDSimulationsSteady2017}.
Thereafter, the hull forces are calculated by subtracting the rudder and propeller forces from the total forces as,
\begin{equation}
    \label{eq:X_H_VCT}
    \left.\begin{aligned}
    X_H = X_D - X_R - X_P \\
    Y_H = Y_D - Y_R \\
    N_H = N_D - N_R
    \end{aligned}\right\}
\end{equation}
% \begin{equation}
%     \label{eq:Y_H_VCT}
%     Y_H = Y_D - Y_R
% \end{equation}
% \begin{equation}
%     \label{eq:N_H_VCT}
%     N_H = N_D - N_R
% \end{equation}
These forces are used together with the hydrodynamic forces applied on the ship hull as in (\autoref{eq:XYN_H}, \autoref{eq:XYN_H_prime}) to define a linear regression problem that is solved with the ordinary least square (OLS) method.

\textbf{\textit{State VCT:}} in addition, some VCT calculations were also conducted for a selection of the states $\pmb{\bm{\upsilon}}$ and inputs $\mathbf{u}$ encountered by the ship during the zigzag tests to check if the fitted maneuvering models predict the same forces as the VCT analysis.

\begin{figure}[h]
     \centering
     \begin{subfigure}[b]{0.49\textwidth}
         \centering
         \includesvg{figures/methodology_VCT_wPCC.phase_plot.svg}
        \caption{wPCC.}
        \label{fig:VCT_phase_plot_wPCC}
     \end{subfigure}
     \hfill
     \begin{subfigure}[b]{0.49\textwidth}
        \centering
        \includesvg{figures/methodology_VCT_optiwise.phase_plot.svg}
        \caption{Optiwise.}
        \label{fig:VCT_phase_plot_optiwise}
     \end{subfigure}
        \caption{Phase plots of the zigzag tests together with the coverage of the VCTs and extra state VCTs.}
        \label{fig:phase_plots}
\end{figure}

% wPCC
\begin{table}[h]
    \centering
    \small
    \caption{Various test scenarios for VCTs of the wPCC.}
    \label{tab:VCT_wPCC}
    \pgfplotstabletypeset[col sep=comma, column type=c, style=string type,
        columns/Test type/.style={column type=l,string type},
        columns/V/.style={column type=c,string type, column name=$V$ [m/s]},
        columns/beta_deg/.style={column type=c,string type, column name=$\beta$ [deg]},
        columns/r/.style={column type=c,string type, column name=$r$ [rad/s]},
        columns/delta_deg/.style={column type=c,string type, column name=$\delta$ [deg]},
        columns/rev/.style={column type=c,string type, column name=rev [1/s]},
        %columns/r/.style={column type=r,fixed,fixed zerofill,precision=2, column name=$r$ [rad/s]},
        %columns/V_R/.style={fixed,fixed zerofill,precision=2, column name=$V_R$ [m/s]},
        %columns/gamma_deg/.style={fixed,fixed zerofill,precision=1, column name=$\gamma$ [deg]},
        %columns/Y_R/.style={fixed,fixed zerofill,precision=1, column name=$Y_R^{VCT}$ [N]},
        %columns/Y_R_MMG/.style={fixed,fixed zerofill,precision=1, column name=$Y_R^{MMG}$ [N]},
        every head row/.style={before row=\hline,after row=\hline},
        every last row/.style={after row=\hline}
    ]{tables/methodology_VCT_wPCC.variations.csv}
\end{table}
%\begin{figure}[h]
%    \includesvg{figures/methodology_VCT_wPCC.phase_plot.svg}
%    \caption{Phase plots of the wPCC zigzag tests together with the coverage of the VCTs and extra state VCTs.}
%    \label{fig:VCT_phase_plot_wPCC}
%\end{figure}

% Optiwise
\begin{table}[h]
    \centering
    \small
    \caption{Various test scenarios for VCTs of the Optiwise.}
    \label{tab:VCT_optiwise}
    \pgfplotstabletypeset[col sep=comma, column type=c, style=string type,
        columns/Test type/.style={column type=l,string type},
        columns/V/.style={column type=c,string type, column name=$V$ [m/s]},
        columns/beta_deg/.style={column type=c,string type, column name=$\beta$ [deg]},
        columns/r/.style={column type=c,string type, column name=$r$ [rad/s]},
        columns/delta_deg/.style={column type=c,string type, column name=$\delta$ [deg]},
        columns/rev/.style={column type=c,string type, column name=rev [1/s]},
        %columns/r/.style={column type=r,fixed,fixed zerofill,precision=2, column name=$r$ [rad/s]},
        %columns/V_R/.style={fixed,fixed zerofill,precision=2, column name=$V_R$ [m/s]},
        %columns/gamma_deg/.style={fixed,fixed zerofill,precision=1, column name=$\gamma$ [deg]},
        %columns/Y_R/.style={fixed,fixed zerofill,precision=1, column name=$Y_R^{VCT}$ [N]},
        %columns/Y_R_MMG/.style={fixed,fixed zerofill,precision=1, column name=$Y_R^{MMG}$ [N]},
        every head row/.style={before row=\hline,after row=\hline},
        every last row/.style={after row=\hline}
    ]{tables/methodology_VCT_optiwise.variations.csv}
\end{table}
%\begin{figure}[h]
%    \includesvg{figures/methodology_VCT_optiwise.phase_plot.svg}
%    \caption{Phase plots of the Optiwise zigzag tests together with the coverage of the VCTs and extra state VCTs.}
%    \label{fig:VCT_phase_plot_optiwise}
%\end{figure}