VCT calculations are conducted by solving a set of static flow calculations with CFD. The VCT test matrix is selected to have a good coverage of the states that the ship will have during the manoeuvre. How the combinations of drift angle and yaw rate have been selected can be seen in the figure below, for the wPCC. 
The total forces from the VCT $X_{VCT}$, $Y_{VCT}$, and $N_{VCT}$ are recalculated in the following way, to obtain the damping forces.
\begin{equation}
    \label{eq:X_D}
    X_{D} = X_{VCT} + Y_{\dot{r}} r^{2} + Y_{\dot{v}} r v
\end{equation}
\begin{equation}
    \label{eq:Y_D}
    Y_{D} = - X_{\dot{u}} r u + Y_{VCT}
\end{equation}
\begin{equation}
    \label{eq:N_D}
    N_{D} = N_{VCT} + X_{\dot{u}} u v - Y_{\dot{r}} r u - Y_{\dot{v}} u v
\end{equation}
The mass $m$ has disappeared from \autoref{eq:F_expanded} to arrive at these expressions, because the ship is not moving in ShipFlow, instead the water is having an either oblique or circular inflow \citep{roychoudhuryCFDSimulationsSteady2017}.
% wPCC
\begin{table}[h]
    \centering
    \small
    \caption{VCT variations for wPCC.}
    \label{tab:inflow_to_rudder_force}
    \pgfplotstabletypeset[col sep=comma, column type=c, style=string type,
        columns/Test type/.style={column type=l,string type},
        columns/V/.style={column type=c,string type, column name=$V$ [m/s]},
        columns/beta_deg/.style={column type=c,string type, column name=$\beta$ [deg]},
        columns/r/.style={column type=c,string type, column name=$r$ [rad/s]},
        columns/delta_deg/.style={column type=c,string type, column name=$\delta$ [deg]},
        columns/rev/.style={column type=c,string type, column name=rev [1/s]},
        %columns/r/.style={column type=r,fixed,fixed zerofill,precision=2, column name=$r$ [rad/s]},
        %columns/V_R/.style={fixed,fixed zerofill,precision=2, column name=$V_R$ [m/s]},
        %columns/gamma_deg/.style={fixed,fixed zerofill,precision=1, column name=$\gamma$ [deg]},
        %columns/Y_R/.style={fixed,fixed zerofill,precision=1, column name=$Y_R^{VCT}$ [N]},
        %columns/Y_R_MMG/.style={fixed,fixed zerofill,precision=1, column name=$Y_R^{MMG}$ [N]},
        every head row/.style={before row=\hline,after row=\hline},
        every last row/.style={after row=\hline}
    ]{tables/methodology_VCT_wPCC.variations.csv}
\end{table}
\begin{figure}[h]
    \includesvg{figures/methodology_VCT_wPCC.phase_plot.svg}
    \caption{Phase plots of the wPCC zigzag tests together with the coverage of the VCTs and extra state VCTs.}
    \label{fig:VCT_phase_plot_wPCC}
\end{figure}

% Optiwise
\begin{table}[h]
    \centering
    \small
    \caption{VCT variations for Optiwise.}
    \label{tab:VCT_optiwise}
    \pgfplotstabletypeset[col sep=comma, column type=c, style=string type,
        columns/Test type/.style={column type=l,string type},
        columns/V/.style={column type=c,string type, column name=$V$ [m/s]},
        columns/beta_deg/.style={column type=c,string type, column name=$\beta$ [deg]},
        columns/r/.style={column type=c,string type, column name=$r$ [rad/s]},
        columns/delta_deg/.style={column type=c,string type, column name=$\delta$ [deg]},
        columns/rev/.style={column type=c,string type, column name=rev [1/s]},
        %columns/r/.style={column type=r,fixed,fixed zerofill,precision=2, column name=$r$ [rad/s]},
        %columns/V_R/.style={fixed,fixed zerofill,precision=2, column name=$V_R$ [m/s]},
        %columns/gamma_deg/.style={fixed,fixed zerofill,precision=1, column name=$\gamma$ [deg]},
        %columns/Y_R/.style={fixed,fixed zerofill,precision=1, column name=$Y_R^{VCT}$ [N]},
        %columns/Y_R_MMG/.style={fixed,fixed zerofill,precision=1, column name=$Y_R^{MMG}$ [N]},
        every head row/.style={before row=\hline,after row=\hline},
        every last row/.style={after row=\hline}
    ]{tables/methodology_VCT_optiwise.variations.csv}
\end{table}
\begin{figure}[h]
    \includesvg{figures/methodology_VCT_optiwise.phase_plot.svg}
    \caption{Phase plots of the Optiwise zigzag tests together with the coverage of the VCTs and extra state VCTs.}
    \label{fig:VCT_phase_plot_optiwise}
\end{figure}