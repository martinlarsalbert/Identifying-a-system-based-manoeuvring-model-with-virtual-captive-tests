
The damping forces and moments are expressed in a modular way, similar to the MMG model \citep{yasukawaIntroductionMMGStandard2015}, as shown in \autoref{eq:X_D}--\autoref{eq:N_D} and \autoref{fig:force_model},
% Components:
\begin{equation}
    \label{eq:X_D}
    X_{D} = X_{H} + X_{P} + X_{R}
\end{equation}
%
\begin{equation}
    \label{eq:Y_D}
    Y_{D} = Y_{H} + Y_{P} + Y_{R} + Y_{RHI}
\end{equation}
%
\begin{equation}
    \label{eq:N_D}
    N_{D} = N_{H} + N_{P} + N_{R} + N_{RHI}
\end{equation}
%
\begin{figure}[h]
    \centering
    \includesvg[width=4cm]{figures/force_model.svg}
    \caption{Modular force components.}
    \label{fig:force_model}
\end{figure}
where subscripts $H$, $P$, $R$, and $RHI$ represent contributions from the hull, propellers, rudders, and rudder hull interaction, respectively. The rudder hull interaction having its own element is a difference from the MMG model.