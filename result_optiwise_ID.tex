Forces predicted with the Optiwise model equipped with either the semi-empirical rudder or the polynomial rudder have been compared with forces from the experiment, estimated by inverse dynamics. Generally, the predicted forces agrees well with the experimental results for the zigzag10/10, as shown in \autoref{fig:ID_optiwise10}. 

As a complement to the inverse dynamics analysis, closed loop simulations were also conducted for the Optiwise. The simulation results can be seen in \autoref{fig:sim_optiwise}. The is a very good agreement between the experimental results and the simulation with the polynomial rudder equipped model. 
\begin{figure}[h]
     \centering
     \begin{subfigure}[b]{\textwidth}
         \centering
         \includesvg{figures/results_optiwise_ID.zigzag 10_10.svg}
        \caption{Forces Optiwise Zigzag10/10 to port.}
        \label{fig:ID_optiwise10}
     \end{subfigure}
     \vfill
     \begin{subfigure}[b]{\textwidth}
         \includesvg{figures/results_optiwise_ID.zigzag 20_20.svg}
        \caption{Forces Optiwise Zigzag20/20 to starboard.}
        \label{fig:ID_optiwise_20}
     \end{subfigure}
        \caption{Comparison between forces during zigzag tests with Optiwise estimated with inverse dynamics from the experiments and predictions with a model equipped with either a polynomial rudder or semi-empirical rudder model. Forces from VCT calculations of some interesting states have also been added.}
        \label{fig:ID_optiwise}
\end{figure}