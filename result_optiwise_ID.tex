Models equipped with the semi-empirical rudder, the MMG rudder, or the polynomial rudder were developed for the Optiwise. Rudder forces were measured during the experiments for this vessel so that a fourth model could be created. This model uses the actual measured rudder force instead of a prediction model -- so that only the hull forces are predicted. 

\autoref{fig:ID_optiwise} shows comparisons for the zigzag10/10 and 20/20 where model force predictions are compared with corresponding values estimated from the experiments with inverse dynamics. VCT calculations were also conducted for some of the states of these maneuvers, shown by the green dots.
There is generally a good agreement for the zigzag10/10, as shown in \autoref{fig:ID_optiwise10}. Deviations were however observed for the sway force $Y_D$ during about 3 seconds after the rudder changes at t=11-14 s, and t=35-38 s, as indicated in the figure. All of the models and the VCT calculations predict a more straight line in the $Y_D$ time series around these deviation points. 
Filtration errors in the EKF were ruled out as a possible explanation to these deviations by conducting alternative analysis with a low-pass filter instead of the EKF. Accelerations were instead calculated with numeric differentiation of the low-pass filtered signals of repeated zigzag10/10 tests as shown in \autoref{fig:lowpass_deviation_points}.

The force predictions for the zigzag20/20 is shown in \autoref{fig:ID_optiwise20}. Similar deviations as for the zigzag10/10 can be observed for $Y_D$ after the rudder transitions (t=12 s, t=33 s, t=64s). The yawing moment $N_D$ is not well predicted for the polynomial rudder in the end of the port turn (t=23-33 s), where the semi-empirical rudder model has much better agreement with the experiments. When the ship starts to turn back to starboard (t=43-62s) only the model that uses the measured rudder forces agrees well with the experiments. This indicates that both the polynomial and semi-empirical rudder models give bad predictions for this situation, where $\beta>0$, $r>0$, and $\delta<0$. 

\begin{figure}[h]
     \centering
     \begin{subfigure}[b]{\textwidth}
         \centering
         \includesvg{figures/results_optiwise_ID.zigzag 10_10.svg}
        \caption{Forces Optiwise Zigzag10/10 to port.}
        \label{fig:ID_optiwise10}
     \end{subfigure}
     \vfill
     \begin{subfigure}[b]{\textwidth}
         \includesvg{figures/results_optiwise_ID.zigzag 20_20.svg}
        \caption{Forces Optiwise Zigzag20/20 to starboard.}
        \label{fig:ID_optiwise20}
     \end{subfigure}
        \caption{Comparison between forces during zigzag tests with Optiwise estimated with inverse dynamics from the experiments and predictions with a model equipped with either a polynomial rudder or semi-empirical rudder model. Forces from VCT calculations of some interesting states have also been added.}
        \label{fig:ID_optiwise}
\end{figure}
\begin{figure}[h]
    \centering
    \includesvg{figures/results_optiwise_deviation_points.lowpass.svg}
    \caption{Inverse dynamics sway force from repeated zigzag10/10 test to port. The accelerations have been estimated with the EKF and also with numeric differentiation of low-pass filtered signals.}
    \label{fig:lowpass_deviation_points}
\end{figure}