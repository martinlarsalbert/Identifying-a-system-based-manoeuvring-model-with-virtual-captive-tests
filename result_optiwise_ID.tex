Another enhancement to the wPCC FRMTs, was the measurement of rudder forces for Optiwise.
The rudder model within the Optiwise simulation model could then be replaced by the actual measured rudder forces, so that the hull prediction model could be assessed in isolation. There was generally good agreement between the predictions and the corresponding inverse dynamics forces for zigzag10/10 (\autoref{fig:ID_measured_rudder_zigzag_10_10}) and zigzag20/20 (\autoref{fig:ID_measured_rudder_zigzag_20_20}).
However, deviations were observed for the sway force $Y_D$ during about 3 seconds after the rudder changes at t = 11--14 s and t = 35--38 s, for the zigzag10/10 and t = 11--14 s, t = 35--38 s, t = 64 s, for the zigzag20/20. The model and state VCT calculations predict a more straight line in the $Y_D$ time series around these deviation points. 
During the work in this article, a reasonable explanation for these deviations has not been found, where filtration errors in the EKF were ruled out as a possible explanation by conducting alternative analysis with a low-pass filter instead of the EKF. Accelerations were instead calculated with numeric differentiation of low-pass filtered signals of repeated zigzag10/10 tests as shown in \autoref{fig:lowpass_deviation_points}.
\begin{figure}[h]
    \centering
    \begin{subfigure}[b]{\textwidth}
        \centering
        \includesvg{figures/results_optiwise_ID.measured_rudder_zigzag 10_10.svg}
        \caption{Zigzag10/10 to port.}
        \label{fig:ID_measured_rudder_zigzag_10_10}
    \end{subfigure}
     \vfill
    \begin{subfigure}[b]{\textwidth}
        \centering
        \includesvg{figures/results_optiwise_ID.measured_rudder_zigzag 20_20.svg}
        \caption{Zigzag20/20 to starboard.}
        \label{fig:ID_measured_rudder_zigzag_20_20}
    \end{subfigure}
    \caption{Inverse dynamics forces during the zigzag tests compared to predictions with the measured rudder model.}
    \label{fig:ID_optiwise20}
\end{figure}
\begin{figure}[h]
    \centering
    \includesvg{figures/results_optiwise_deviation_points.lowpass.svg}
    \caption{Inverse dynamics sway force from repeated zigzag10/10 test to port. The accelerations have been estimated with the EKF and also with numeric differentiation of low-pass filtered signals.}
    \label{fig:lowpass_deviation_points}
\end{figure}
\FloatBarrier

Given the good results with the measured rudder model, focus was on finding a good rudder force prediction model for Optiwise. The MMG original model and the MMG quadratic model -- proposed in this paper -- where fitted to the rudder forces from the VCT data. These models predicted similar forces during the zigzag tests which agreed quite well with the measured forces, especially for the zigzag20/20 test, as shown in \autoref{fig:ID_measured_rudder_zigzag_10_10} and \autoref{fig:ID_measured_rudder_zigzag_20_20}.
Similar agreement achieved by the measured rudder model could now be achieved by the models equipped with MMG rudder models, as shown in \autoref{fig:ID_optiwise20}.


\begin{figure}[h]
    \centering
    \begin{subfigure}[b]{\textwidth}
        \centering
        \includesvg{figures/results_optiwise_ID.rudder_forces_zigzag 10_10.svg}
        \caption{Zigzag10/10 to port.}
        \label{fig:ID_measured_rudder_zigzag_10_10}
    \end{subfigure}
     \vfill
    \begin{subfigure}[b]{\textwidth}
        \centering
        \includesvg{figures/results_optiwise_ID.rudder_forces_zigzag 20_20.svg}
        \caption{Zigzag20/20 to starboard.}
        \label{fig:ID_measured_rudder_zigzag_20_20}
    \end{subfigure}
    \caption{Rudder forces during the zigzag tests compared to predictions with the MMG models.}
    \label{fig:ID_optiwise20}
\end{figure}


\begin{figure}[h]
    \centering
    \begin{subfigure}[b]{\textwidth}
        \centering
        \includesvg{figures/results_optiwise_ID.zigzag 10_10.svg}
        \caption{Zigzag10/10 to port.}
        \label{fig:ID_MMG_zigzag_10_10}
    \end{subfigure}
     \vfill
    \begin{subfigure}[b]{\textwidth}
        \centering
        \includesvg{figures/results_optiwise_ID.zigzag 20_20.svg}
        \caption{Zigzag20/20 to starboard.}
        \label{fig:ID_MMG_zigzag_20_20}
    \end{subfigure}
    \caption{Inverse dynamics forces during the zigzag tests compared to predictions with the MMG models.}
    \label{fig:ID_optiwise20}
\end{figure}


