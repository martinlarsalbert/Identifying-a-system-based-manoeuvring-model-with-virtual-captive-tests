The goodness of fit of the identified prediction models for Optiwise was investigated by visual inspection as shown in \autoref{fig:drift_angle_optiwise} to \autoref{fig:rudder_angle_compare_optiwise}. The total hydrodynamic damping forces (subscript D), hull forces (H), rudder forces (R) and propeller (P)   

The force components from the Optiwise VCT and corresponding predictions with the identified model for the drift angle tests are shown in \autoref{fig:drift_angle_wPCC}. The total damping forces $X_D$, $Y_D$, and $N_D$ are the sum of the components from the hull (H), rudder (R) and propellers (P). Both the hull and the rudder contribute to the total sway force as shown in \autoref{fig:drift_angle_Y}. The hull yawing moment is close to zero for drift angles up to 10 degrees as shown in \autoref{fig:drift_angle_N}. This means that the total hull yawing moment acting on the ship, is only generated by the inviscid Munk moment -- which is not included in $N_H$. A nonlinear viscous contribution is present for larger drift angles. The rudders are however the main contributors to the viscous yawing moment.   
%Drift angle
\begin{figure}[h]
     \centering
     \begin{subfigure}[b]{0.32\textwidth}
         \centering
         \includesvg{figures/results_optiwise_VCT.drift_angle_X.svg}
        \caption{Surge force.}
        \label{fig:drift_angle_X_optiwise}
     \end{subfigure}
     \hfill
     \begin{subfigure}[b]{0.32\textwidth}
         \centering
         \includesvg{figures/results_optiwise_VCT.drift_angle_Y.svg}
        \caption{Sway force.}
        \label{fig:drift_angle_Y_optiwise}
     \end{subfigure}
     \hfill
     \begin{subfigure}[b]{0.32\textwidth}
         \centering
         \includesvg{figures/results_optiwise_VCT.drift_angle_N.svg}
        \caption{Yawing moment.}
        \label{fig:drift_angle_N_optiwise}
     \end{subfigure}
    \caption{Optiwise drift angle tests from VCT (dots) and predictions (lines).}
    \label{fig:drift_angle_optiwise}
\end{figure}
%Circle
\begin{figure}[h]
     \centering
     \begin{subfigure}[b]{0.32\textwidth}
         \centering
         \includesvg{figures/results_optiwise_VCT.circle_X.svg}
        \caption{Surge force.}
        \label{fig:drift_angle_X_optiwise}
     \end{subfigure}
     \hfill
     \begin{subfigure}[b]{0.32\textwidth}
         \centering
         \includesvg{figures/results_optiwise_VCT.circle_Y.svg}
        \caption{Sway force.}
        \label{fig:drift_angle_Y_optiwise}
     \end{subfigure}
     \hfill
     \begin{subfigure}[b]{0.32\textwidth}
         \centering
         \includesvg{figures/results_optiwise_VCT.circle_N.svg}
        \caption{Yawing moment.}
        \label{fig:drift_angle_N_optiwise}
     \end{subfigure}
    \caption{Optiwise circle tests from VCT (dots) and predictions (lines).}
    \label{fig:circle_optiwise}
\end{figure}
%Circle + drift
\begin{figure}[h]
     \centering
     \begin{subfigure}[b]{0.49\textwidth}
         \centering
         \includesvg{figures/results_optiwise_VCT.Y_H.svg}
        \caption{Sway force.}
        \label{fig:circle_drift_Y_H_optiwise}
     \end{subfigure}
     \hfill
     \begin{subfigure}[b]{0.49\textwidth}
         \centering
         \includesvg{figures/results_optiwise_VCT.N_H.svg}
        \caption{Yawing moment.}
        \label{fig:circle_drift_N_H_optiwise}
     \end{subfigure}
    \caption{Hull forces during the circle and drift variations, VCT (dots), fitted model (surface).}
    \label{fig:circle_drift_optiwise}
\end{figure}
%Rudder angle
\begin{figure}[h]
     \centering
     \begin{subfigure}[b]{0.32\textwidth}
         \centering
         \includesvg{figures/results_optiwise_VCT.rudder_angle_X.svg}
        \caption{Surge force.}
        \label{fig:rudder_angle_X_optiwise}
     \end{subfigure}
     \hfill
     \begin{subfigure}[b]{0.32\textwidth}
         \centering
         \includesvg{figures/results_optiwise_VCT.rudder_angle_Y.svg}
        \caption{Sway force.}
        \label{fig:rudder_angle_Y_optiwise}
     \end{subfigure}
     \hfill
     \begin{subfigure}[b]{0.32\textwidth}
         \centering
         \includesvg{figures/results_optiwise_VCT.rudder_angle_N.svg}
        \caption{Yawing moment.}
        \label{fig:rudder_angle_N_optiwise}
     \end{subfigure}
    \caption{Optiwise rudder angle tests from VCT (dots) and predictions (lines).}
    \label{fig:rudder_angle_optiwise}
\end{figure}
%beta_R
\begin{figure}[h]
     \centering
     \begin{subfigure}[b]{0.49\textwidth}
         \centering
         \includesvg{figures/results_optiwise_VCT.Y_R_MMG_original.svg}
        \caption{Original MMG rudder model.}
        \label{fig:Y_R_MMG_original}
     \end{subfigure}
     \hfill
     \begin{subfigure}[b]{0.49\textwidth}
         \centering
         \includesvg{figures/results_optiwise_VCT.Y_R_MMG_quadratic.svg}
        \caption{Modified quadratic MMG rudder model.}
        \label{fig:Y_R_MMG_quadratic}
     \end{subfigure}
    \caption{Rudder force during the VCT tests as function of the effective inflow angle for the original MMG model and the modified quadratic MMG model.}
    \label{fig:circle_drift}
\end{figure}
%gamma_0, thrust_variation
\begin{figure}[h]
     \centering
     \begin{subfigure}[b]{0.49\textwidth}
         \centering
         \includesvg{figures/results_optiwise_VCT.rudder_angle.svg}
        \caption{Rudder angle variation.}
        \label{fig:rudder_angle_compare_optiwise}
     \end{subfigure}
     \hfill
     \begin{subfigure}[b]{0.49\textwidth}
         \centering
         \includesvg{figures/results_optiwise_VCT.thrust_variation.svg}
        \caption{Thrust variation.}
        \label{fig:thrust_variation_optiwise}
     \end{subfigure}
    \caption{Rudder angle variation and thrust variation.}
    \label{fig:rudder_angle_compare_optiwise}
\end{figure}