Rudder force predictions with the MMG original rudder model and the modified MMG quadratic model were compared to the VCT data for the rudder angle variation as shown in \autoref{fig:rudder_angle_compare_optiwise}. The newly added initial inflow angle parameter $\gamma_0$ (see \autoref{eq:gamma_R2}) allows the MMG quadratic  model to have a better fit to the VCT data. The difference between the models is however less obvious when looking at the thrust variation tests (\autoref{fig:thrust_variation_optiwise}).
%gamma_0, thrust_variation
\begin{figure}[h]
     \centering
     \begin{subfigure}[b]{0.49\textwidth}
         \centering
         \includesvg{figures/results_optiwise_VCT.rudder_angle.svg}
        \caption{Rudder angle variation.}
        \label{fig:rudder_angle_compare_optiwise}
     \end{subfigure}
     \hfill
     \begin{subfigure}[b]{0.49\textwidth}
         \centering
         \includesvg{figures/results_optiwise_VCT.thrust_variation.svg}
        \caption{Thrust variation at \textpm 10 degrees rudder angle.}
        \label{fig:thrust_variation_optiwise}
     \end{subfigure}
    \caption{Rudder angle variation and thrust variation.}
    \label{fig:rudder_angle_compare_optiwise_all}
\end{figure}

The newly added quadratic relationship for the flow straightening coefficient $\gamma_R$ (\autoref{eq:gamma_R2}) in the MMG quadratic rudder model has a better fit to the VCT data than the MMG original rudder model as shown in \autoref{fig:MMG_quadratic}.
%beta_R
\begin{figure}[h]
     \centering
     \begin{subfigure}[b]{0.49\textwidth}
         \centering
         \includesvg{figures/results_optiwise_VCT.Y_R_MMG_original.svg}
        \caption{Original MMG rudder model.}
        \label{fig:Y_R_MMG_original}
     \end{subfigure}
     \hfill
     \begin{subfigure}[b]{0.49\textwidth}
         \centering
         \includesvg{figures/results_optiwise_VCT.Y_R_MMG_quadratic.svg}
        \caption{Modified quadratic MMG rudder model.}
        \label{fig:Y_R_MMG_quadratic}
     \end{subfigure}
    \caption{Rudder force during the VCT tests as function of the effective inflow angle for the original MMG model and the modified quadratic MMG model.}
    \label{fig:MMG_quadratic}
\end{figure}

\autoref{fig:rudder_angle_Y_optiwise} shows that side force is generated both on the rudder and on the hull surface during the rudder angle variation, which is well predicted by the model through the rudder hull interaction coefficients $x_R$ and $a_R$. 

The model predicts zero rudder drag when the rudder angle is zero at straight ahead condition as shown in \autoref{fig:rudder_angle_X_optiwise}. This is because the MMG rudder model has no base drag coefficient like the semi-empirical rudder model for the wPCC (see \autoref{fig:rudder_angle_X_wPCC}).

Similar comparisons are shown for the drift angle tests in \autoref{fig:drift_angle_X_optiwise} -- \autoref{fig:drift_angle_N_optiwise} and the circle tests in \autoref{fig:circle_X_optiwise} -- \autoref{fig:circle_N_optiwise}. 
\begin{figure}[h]
    \centering
    %Rudder angle
    \begin{subfigure}[b]{0.32\textwidth}
         \centering
         \includesvg{figures/results_optiwise_VCT.rudder_angle_X.svg}
        \caption{Rudder angle X.}
        \label{fig:rudder_angle_X_optiwise}
    \end{subfigure}
    \hfill
    \begin{subfigure}[b]{0.32\textwidth}
        \centering
        \includesvg{figures/results_optiwise_VCT.rudder_angle_Y.svg}
       \caption{Rudder angle Y.}
       \label{fig:rudder_angle_Y_optiwise}
    \end{subfigure}
    \hfill
    \begin{subfigure}[b]{0.32\textwidth}
        \centering
        \includesvg{figures/results_optiwise_VCT.rudder_angle_N.svg}
       \caption{Rudder angle N.}
       \label{fig:rudder_angle_N_optiwise}
    \end{subfigure}

    \vfill
    %Drift angle
    \begin{subfigure}[b]{0.32\textwidth}
        \centering
        \includesvg{figures/results_optiwise_VCT.drift_angle_X.svg}
       \caption{Drift angle X.}
       \label{fig:drift_angle_X_optiwise}
    \end{subfigure}
    \hfill
    \begin{subfigure}[b]{0.32\textwidth}
        \centering
        \includesvg{figures/results_optiwise_VCT.drift_angle_Y.svg}
       \caption{Drift angle Y.}
       \label{fig:drift_angle_Y_optiwise}
    \end{subfigure}
    \hfill
    \begin{subfigure}[b]{0.32\textwidth}
        \centering
        \includesvg{figures/results_optiwise_VCT.drift_angle_N.svg}
       \caption{Drift angle N.}
       \label{fig:drift_angle_N_optiwise}
    \end{subfigure}
    
    \vfill
    %Circle
    \begin{subfigure}[b]{0.32\textwidth}
        \centering
        \includesvg{figures/results_optiwise_VCT.circle_X.svg}
       \caption{Circle X.}
       \label{fig:circle_X_optiwise}
    \end{subfigure}
    \hfill
    \begin{subfigure}[b]{0.32\textwidth}
        \centering
        \includesvg{figures/results_optiwise_VCT.circle_Y.svg}
       \caption{Circle Y.}
       \label{fig:circle_Y_optiwise}
    \end{subfigure}
    \hfill
    \begin{subfigure}[b]{0.32\textwidth}
        \centering
        \includesvg{figures/results_optiwise_VCT.circle_N.svg}
       \caption{Circle N.}
       \label{fig:circle_N_optiwise}
    \end{subfigure}
    
    \caption{Optiwise VCT (dots) and predictions (lines).}
    \label{fig:VCT_optiwise}
\end{figure}

The coupling terms $Y_{vrr}$,$Y_{vvr}$,$N_{vrr}$, and $N_{vvr}$ in the hull force model (\autoref{eq:Y_H}, \autoref{eq:N_H}) where fitted from the circle and drift variations. These coupling terms are important as shown by the comparison with/without them in \autoref{fig:circle_drift_optiwise}.
%Circle + drift
\begin{figure}[h]
     \centering
     \begin{subfigure}[b]{0.49\textwidth}
         \centering
         \includesvg{figures/results_optiwise_VCT.Y_H.svg}
        \caption{Sway force.}
        \label{fig:circle_drift_Y_H_optiwise}
     \end{subfigure}
     \hfill
     \begin{subfigure}[b]{0.49\textwidth}
         \centering
         \includesvg{figures/results_optiwise_VCT.Y_H_no_coupling.svg}
        \caption{Sway force no coupling.}
        \label{fig:circle_drift_Y_H_no_coupling_optiwise}
     \end{subfigure}

     \vfill
     \begin{subfigure}[b]{0.49\textwidth}
         \centering
         \includesvg{figures/results_optiwise_VCT.N_H.svg}
        \caption{Yawing moment.}
        \label{fig:circle_drift_N_H_optiwise}
     \end{subfigure}
     \hfill
     \begin{subfigure}[b]{0.49\textwidth}
         \centering
         \includesvg{figures/results_optiwise_VCT.N_H_no_coupling.svg}
        \caption{Yawing moment no coupling.}
        \label{fig:circle_drift_N_H_no_coupling_optiwise}
     \end{subfigure}
     
    \caption{Optiwise hull forces during the circle and drift variations with/without the coupling terms, VCT (dots), fitted model (surface).}
    \label{fig:circle_drift_optiwise}
\end{figure}

The identified hull parameters for both wPCC and Optiwise are shown in \autoref{tab:parameters}. The added masses from the pure yaw and pure sway tests are shown in \autoref{tab:added_masses}.
\begin{table}[h]
    \centering
    \caption{Identified hull coefficients in prime system units.}
    \label{tab:parameters}
    \pgfplotstabletypeset[col sep=comma, column type=r,
        columns/Coefficient/.style={column type=l,string type},
    every head row/.style={before row=\hline,after row=\hline},
    every last row/.style={after row=\hline}
    ]{tables/result_models.parameters.csv}
\end{table}
\begin{table}[h]
    \centering
    \caption{Added masses in prime system units times 1000.}
    \label{tab:added_masses}
    \pgfplotstabletypeset[col sep=comma, column type=c,
    columns/Ship/.style={column type=l, string type},
    columns/Xudot/.style={column name=$X_{\dot{u}}$},
    columns/Yvdot/.style={column name=$Y_{\dot{v}}$},
    columns/Yrdot/.style={column name=$Y_{\dot{r}}$},
    columns/Nvdot/.style={column name=$N_{\dot{v}}$},
    columns/Nrdot/.style={column name=$N_{\dot{r}}$},
    every head row/.style={before row=\hline,after row=\hline},
    every last row/.style={after row=\hline}
    ]{tables/result_models.added_masses.csv}
\end{table}
\begin{table}[h]
    \centering
    \caption{wPCC Semi-empirical rudder parameters (SI units) in model scale.}
    \label{tab:wPCC_other_parameters}
    \pgfplotstabletypeset[col sep=comma, column type=r,
    columns/Parameter/.style={column type=l,string type,column name=Par.},
    columns/Parameter1/.style={column type=l,string type, column name=Par.},
    columns/Parameter2/.style={column type=l,string type, column name=Par.},
    columns/Value/.style={column type=r, column name=~},
    columns/Value1/.style={column type=r, column name=~},
    columns/Value2/.style={column type=r, column name=~},
    columns/Unit/.style={column type=l,string type, column name=~},
    columns/Unit1/.style={column type=l,string type, column name=~},
    columns/Unit2/.style={column type=l,string type, column name=~},
    every head row/.style={before row=\hline,after row=\hline},
    every last row/.style={after row=\hline}
    ]{tables/result_models.wPCC_other_parameters.csv}
\end{table}

\begin{table}[h]
    \centering
    \caption{Optiwise MMG rudder parameters.}
    \label{tab:optiwise_other_parameters}
    \pgfplotstabletypeset[col sep=comma, column type=r,
    columns/Parameter/.style={column type=l,string type,column name=Par.},
    columns/MMG original/.style={column type=r},
    columns/MMG quadratic.style={column type=r},
    every head row/.style={before row=\hline,after row=\hline},
    every last row/.style={after row=\hline}
    ]{tables/result_models.optiwise_other_parameters.csv}
\end{table}