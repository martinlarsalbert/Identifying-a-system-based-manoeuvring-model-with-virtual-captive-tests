Rudder force predictions with the MMG original rudder model and the modified MMG quadratic model are compared with the VCT data for the rudder angle variation as shown in \autoref{fig:rudder_angle_compare_optiwise}. The newly added initial inflow angle parameter $\gamma_0$ (see \autoref{eq:gamma_R2}) allows the MMG quadratic  model to have a better fit to the VCT data. The difference between the models is however less obvious when looking at the thrust variation tests (\autoref{fig:thrust_variation_optiwise}).
%gamma_0, thrust_variation
\begin{figure}[h]
     \centering
     \begin{subfigure}[b]{0.49\textwidth}
         \centering
         \includesvg{figures/results_optiwise_VCT.rudder_angle.svg}
        \caption{Rudder angle variation.}
        \label{fig:rudder_angle_compare_optiwise}
     \end{subfigure}
     \hfill
     \begin{subfigure}[b]{0.49\textwidth}
         \centering
         \includesvg{figures/results_optiwise_VCT.thrust_variation.svg}
        \caption{Thrust variation at \textpm 10 degrees rudder angle.}
        \label{fig:thrust_variation_optiwise}
     \end{subfigure}
    \caption{Rudder angle variation and thrust variation.}
    \label{fig:rudder_angle_compare_optiwise_all}
\end{figure}

The newly added quadratic relationship for the flow straightening coefficient $\gamma_R$ (\autoref{eq:gamma_R2}) in the MMG quadratic rudder model has a better fit to the VCT data than the MMG original rudder model as shown in \autoref{fig:MMG_quadratic}.
%beta_R
\begin{figure}[h]
     \centering
     \begin{subfigure}[b]{0.49\textwidth}
         \centering
         \includesvg{figures/results_optiwise_VCT.Y_R_MMG_original.svg}
        \caption{Original MMG rudder model.}
        \label{fig:Y_R_MMG_original}
     \end{subfigure}
     \hfill
     \begin{subfigure}[b]{0.49\textwidth}
         \centering
         \includesvg{figures/results_optiwise_VCT.Y_R_MMG_quadratic.svg}
        \caption{Modified quadratic MMG rudder model.}
        \label{fig:Y_R_MMG_quadratic}
     \end{subfigure}
    \caption{Rudder force during the VCT tests as function of the effective inflow angle for the original MMG model and the modified quadratic MMG model.}
    \label{fig:MMG_quadratic}
\end{figure}

\autoref{fig:rudder_angle_Y_optiwise} shows that side force is generated both on the rudder and on the hull surface during the rudder angle variation, which is well predicted by the model predictions through the rudder hull interaction coefficients $x_R$ and $a_R$. 

The model predicts zero rudder drag when the rudder angle is zero at straight ahead condition as shown in \label{fig:rudder_angle_X_optiwise}. This is because the MMG rudder model has no base drag coefficient, which is a simplification compared to the VCT data.
%Rudder angle
\begin{figure}[h]
     \centering
     \begin{subfigure}[b]{0.32\textwidth}
         \centering
         \includesvg{figures/results_optiwise_VCT.rudder_angle_X.svg}
        \caption{Surge force.}
        \label{fig:rudder_angle_X_optiwise}
     \end{subfigure}
     \hfill
     \begin{subfigure}[b]{0.32\textwidth}
         \centering
         \includesvg{figures/results_optiwise_VCT.rudder_angle_Y.svg}
        \caption{Sway force.}
        \label{fig:rudder_angle_Y_optiwise}
     \end{subfigure}
     \hfill
     \begin{subfigure}[b]{0.32\textwidth}
         \centering
         \includesvg{figures/results_optiwise_VCT.rudder_angle_N.svg}
        \caption{Yawing moment.}
        \label{fig:rudder_angle_N_optiwise}
     \end{subfigure}
    \caption{Optiwise rudder angle tests from VCT (dots) and predictions (lines).}
    \label{fig:rudder_angle_optiwise}
\end{figure}

Similar comparisons are shown for the drift angle tests in \autoref{fig:drift_angle_optiwise} and the circle tests in \autoref{fig:circle_optiwise}. 
%Drift angle
\begin{figure}[h]
     \centering
     \begin{subfigure}[b]{0.32\textwidth}
         \centering
         \includesvg{figures/results_optiwise_VCT.drift_angle_X.svg}
        \caption{Surge force.}
        \label{fig:drift_angle_X_optiwise}
     \end{subfigure}
     \hfill
     \begin{subfigure}[b]{0.32\textwidth}
         \centering
         \includesvg{figures/results_optiwise_VCT.drift_angle_Y.svg}
        \caption{Sway force.}
        \label{fig:drift_angle_Y_optiwise}
     \end{subfigure}
     \hfill
     \begin{subfigure}[b]{0.32\textwidth}
         \centering
         \includesvg{figures/results_optiwise_VCT.drift_angle_N.svg}
        \caption{Yawing moment.}
        \label{fig:drift_angle_N_optiwise}
     \end{subfigure}
    \caption{Optiwise drift angle tests from VCT (dots) and predictions (lines).}
    \label{fig:drift_angle_optiwise}
\end{figure}
%Circle
\begin{figure}[h]
     \centering
     \begin{subfigure}[b]{0.32\textwidth}
         \centering
         \includesvg{figures/results_optiwise_VCT.circle_X.svg}
        \caption{Surge force.}
        \label{fig:drift_angle_X_optiwise}
     \end{subfigure}
     \hfill
     \begin{subfigure}[b]{0.32\textwidth}
         \centering
         \includesvg{figures/results_optiwise_VCT.circle_Y.svg}
        \caption{Sway force.}
        \label{fig:drift_angle_Y_optiwise}
     \end{subfigure}
     \hfill
     \begin{subfigure}[b]{0.32\textwidth}
         \centering
         \includesvg{figures/results_optiwise_VCT.circle_N.svg}
        \caption{Yawing moment.}
        \label{fig:drift_angle_N_optiwise}
     \end{subfigure}
    \caption{Optiwise circle tests from VCT (dots) and predictions (lines).}
    \label{fig:circle_optiwise}
\end{figure}
%Circle + drift
\begin{figure}[h]
     \centering
     \begin{subfigure}[b]{0.49\textwidth}
         \centering
         \includesvg{figures/results_optiwise_VCT.Y_H.svg}
        \caption{Sway force.}
        \label{fig:circle_drift_Y_H_optiwise}
     \end{subfigure}
     \hfill
     \begin{subfigure}[b]{0.49\textwidth}
         \centering
         \includesvg{figures/results_optiwise_VCT.N_H.svg}
        \caption{Yawing moment.}
        \label{fig:circle_drift_N_H_optiwise}
     \end{subfigure}
    \caption{Hull forces during the circle and drift variations, VCT (dots), fitted model (surface).}
    \label{fig:circle_drift_optiwise}
\end{figure}

