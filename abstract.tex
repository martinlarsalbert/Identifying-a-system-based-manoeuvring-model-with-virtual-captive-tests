Ships with wind-assisted propulsion systems (WAPS) should be equipped with large rudders to compensate for WAPS-induced drifting forces. The WAPS also significantly affects the effectiveness of mathematical models used to describe the ship's maneuvering characteristics. In this study, the Maneuvering Modeling Group (MMG) is chosen and various approaches are proposed to identify different parameters within the MMG model. First, the ship's hydrodynamic coefficients, such as mass, added mass, and center of gravity, are estimated from either model tests or hydrodynamic analysis. The hydrodynamic forces and drifting forces are proposed to be analyzed by virtual captive tests (VCT). The key part of identifying MMG models for ships with WAPS is to obtain accurate rudder forces. Various explicit semi-empirical formulas are proposed to approximate the rudder forces when experimental data on the rudder forces is not available. Based on the experimental tests with rudder forces measured, the quadratic semi-empirical models are found to perfectly describe such forces. Finally, the other parameters in the MMG model are estimated by the inverse dynamics based on the series of maneuvering tests. In this study, two ships designed with WAPS, i.e., wPCC and Optiwise, are used to validate the proposed method based on their experimental model tests. For the wPCC ship with double rudders, large discrepancies are observed between the close loop maneuvering prediction by the identified model and tests, due to the incapability of VCTs to model rudder forces, and asymmetric patterns of flows on the two rudders. When the measurement of rudder forces is available for the Optiwise ship, the MMG model identified by the proposed method can perfectly predict the ship’s maneuvering motions compared to the tests.
% Move 1 - Background/introduction/situation

% Move 2 - Present research/purpose

% Move 3 - Methods/materials/subjects/procedures

% Move 4 - Results/findings

% Move 5 - Discussion/conclusion/significance