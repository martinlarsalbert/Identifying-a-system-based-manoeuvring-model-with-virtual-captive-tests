% Move 1 - Background/introduction/situation
Ships with wind-assisted propulsion systems (WAPS) are often equipped with large rudders to compensate for WAPS-induced drifting forces. The WAPS also significantly affects the effectiveness of mathematical models used to describe the ship's maneuvering characteristics. In this study, a modular manoeuvring model is proposed to improve the original MMG model to produce accurate manoeuvring simulations for ships with WAPS ships. 
% Move 3 - Methods/materials/subjects/procedures
Methods of virtual captive tests (VCT) to recreate the forces acting on WAPS ships during free-running model tests (FRMT) in motor mode are proposed to identify all the parameters in the modular model. 
The hydrodynamic damping coefficients within the model are determined with linear regression of the VCT data. The added masses are then determined from pure yaw and pure sway simulations with a fully nonlinear potential flow (FNPF) panel method.
% Move 4 - Results/findings
Two ships designed for WAPS - wPCC and Optiwise - are used to validate the proposed method based on the inverse dynamics of their experimental model tests. The wPCC is equipped with a semi-empirical rudder that has been previously shown to work well for this twin-rudder ship. The Optiwise single rudder is modeled with a new quadratic version of the MMG rudder model, proposed in this paper. 
It is concluded that inverse dynamics analysis together with state VCTs is an efficient way to analyze the models, and the manoeuvring model can be efficiently identified when correct VCTs are used in the proposed method. However, the inverse dynamics analysis also revealed that there could be an error in the wPCC VCT data due to false assumptions about wave generation and roll motion. The Optiwise test case, where these assumptions should be more valid, had much better agreement with the FRMT inverse dynamics. 

% Move 5 - Discussion/conclusion/significance
%This paper has proposed a new version of the MMG model that produced accurate simulations when identified on correct VCT data. .

%    

%For the wPCC ship with double rudders, large discrepancies were observed between the closed loop maneuvering predictions by the identified model and tests, due to the incapability of VCTs to model rudder forces, and asymmetric patterns of flows on the two rudders. When the measurement of rudder forces is available for the Optiwise ship, the MMG model identified by the proposed method can perfectly predict the ship’s maneuvering motions compared to the tests.
