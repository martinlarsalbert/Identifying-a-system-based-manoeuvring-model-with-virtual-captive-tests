Ships with wind-assisted ship propulsion (WASP) should be equipped with large rudders to compensate for the WASP induced drifting forces. The WASP will significantly affect the effectiveness for the identification of mathematical models used to describe the maneuvering characteristics of the ship. In this study, Maneuvering Modeling Group (MMG) is chosen to model ship maneuvering characteristics, while various approaches are proposed to identify different parameters within the MMG model. First, the ship hydrodynamic coefficients in the maneuvering motion equations, such as mass, added mass and center of gravity, are estimated from either model tests or hydrodynamic analysis. The hydrodynamic forces acting on the ship hull and drifting forces are proposed to be estimated by VCTs. The key part for reliable MMG models of ships with WASP is to obtain accurate rudder forces. Various explicit semi-empirical formulas were proposed to approximate the rudder forces when experimental data of the rudder forces are not available. Based on the experimental tests with rudder forces measured, the quadratic semi-empirical models can almost perfectly describe such forces. Finally, the other parameters in the MMG model are estimated by the inverse dynamics based on the series of maneuvering tests. Two ships designed with WASP, named as wPCC and Optiwise, are used in this study to validate the proposed method based on their experimental model tests. For the wPCC ship with double rudders, large discrepancies are observed between the close loop maneuvering prediction by the identified model and tests, due to the incapability of VCT to model rudder forces, and asymmetric patterns of flows on the two rudders. While when the measurement of rudder forces is available, the MMG model identified by the proposed method can perfectly predict the ship’s maneuvering motions in comparison with the tests.
% Move 1 - Background/introduction/situation

% Move 2 - Present research/purpose

% Move 3 - Methods/materials/subjects/procedures

% Move 4 - Results/findings

% Move 5 - Discussion/conclusion/significance