Instead of using the the equation of motion (\autoref{eq:acc}) to predict the acceleration $\pmb{\bm{\dot{\upsilon}}}$, this equation can instead be inverted to describe the inverse dynamics of the ship as shown in \autoref{eq:ID_X}-\autoref{eq:ID_N}.
\begin{equation}
    \label{eq:ID_X}
    The forces acting on the ship during a maneuver can be estimated with inverse dynamics from the equation of motion (\autoref{eq:eom}) when the mass matrix $\mathbf{M}$ and the acceleration vector $\pmb{\bm{\dot{\upsilon}}}$ are known. The hydrodynamic damping forces can be calculated by inserting the total force $\mathbf{F}$ from \autoref{eq:F_expanded} into \autoref{eq:eom} and then solve for $X_D$, $Y_D$, and $N_D$ as shown in \autoref{eq:ID_X}-\autoref{eq:ID_N}.
\begin{equation}
    \label{eq:ID_X}
    The forces acting on the ship during a maneuver can be estimated with inverse dynamics from the equation of motion (\autoref{eq:eom}) when the mass matrix $\mathbf{M}$ and the acceleration vector $\pmb{\bm{\dot{\upsilon}}}$ are known. The hydrodynamic damping forces can be calculated by inserting the total force $\mathbf{F}$ from \autoref{eq:F_expanded} into \autoref{eq:eom} and then solve for $X_D$, $Y_D$, and $N_D$ as shown in \autoref{eq:ID_X}-\autoref{eq:ID_N}.
\begin{equation}
    \label{eq:ID_X}
    The forces acting on the ship during a maneuver can be estimated with inverse dynamics from the equation of motion (\autoref{eq:eom}) when the mass matrix $\mathbf{M}$ and the acceleration vector $\pmb{\bm{\dot{\upsilon}}}$ are known. The hydrodynamic damping forces can be calculated by inserting the total force $\mathbf{F}$ from \autoref{eq:F_expanded} into \autoref{eq:eom} and then solve for $X_D$, $Y_D$, and $N_D$ as shown in \autoref{eq:ID_X}-\autoref{eq:ID_N}.
\begin{equation}
    \label{eq:ID_X}
    \input{equations/methodology_ID.X_D}
\end{equation}
\begin{equation}
    \label{eq:ID_Y}
    \input{equations/methodology_ID.Y_D}
\end{equation}
\begin{equation}
    \label{eq:ID_N}
    \input{equations/methodology_ID.N_D}
\end{equation}
These expressions are used to estimate the forces acting on the ship during the FRMTs, which is then compared to the model force predictions. This is a more informative way to assess the model, than to compare its simulations with the FRMTs results.
\end{equation}
\begin{equation}
    \label{eq:ID_Y}
    The forces acting on the ship during a maneuver can be estimated with inverse dynamics from the equation of motion (\autoref{eq:eom}) when the mass matrix $\mathbf{M}$ and the acceleration vector $\pmb{\bm{\dot{\upsilon}}}$ are known. The hydrodynamic damping forces can be calculated by inserting the total force $\mathbf{F}$ from \autoref{eq:F_expanded} into \autoref{eq:eom} and then solve for $X_D$, $Y_D$, and $N_D$ as shown in \autoref{eq:ID_X}-\autoref{eq:ID_N}.
\begin{equation}
    \label{eq:ID_X}
    \input{equations/methodology_ID.X_D}
\end{equation}
\begin{equation}
    \label{eq:ID_Y}
    \input{equations/methodology_ID.Y_D}
\end{equation}
\begin{equation}
    \label{eq:ID_N}
    \input{equations/methodology_ID.N_D}
\end{equation}
These expressions are used to estimate the forces acting on the ship during the FRMTs, which is then compared to the model force predictions. This is a more informative way to assess the model, than to compare its simulations with the FRMTs results.
\end{equation}
\begin{equation}
    \label{eq:ID_N}
    The forces acting on the ship during a maneuver can be estimated with inverse dynamics from the equation of motion (\autoref{eq:eom}) when the mass matrix $\mathbf{M}$ and the acceleration vector $\pmb{\bm{\dot{\upsilon}}}$ are known. The hydrodynamic damping forces can be calculated by inserting the total force $\mathbf{F}$ from \autoref{eq:F_expanded} into \autoref{eq:eom} and then solve for $X_D$, $Y_D$, and $N_D$ as shown in \autoref{eq:ID_X}-\autoref{eq:ID_N}.
\begin{equation}
    \label{eq:ID_X}
    \input{equations/methodology_ID.X_D}
\end{equation}
\begin{equation}
    \label{eq:ID_Y}
    \input{equations/methodology_ID.Y_D}
\end{equation}
\begin{equation}
    \label{eq:ID_N}
    \input{equations/methodology_ID.N_D}
\end{equation}
These expressions are used to estimate the forces acting on the ship during the FRMTs, which is then compared to the model force predictions. This is a more informative way to assess the model, than to compare its simulations with the FRMTs results.
\end{equation}
These expressions are used to estimate the forces acting on the ship during the FRMTs, which is then compared to the model force predictions. This is a more informative way to assess the model, than to compare its simulations with the FRMTs results.
\end{equation}
\begin{equation}
    \label{eq:ID_Y}
    The forces acting on the ship during a maneuver can be estimated with inverse dynamics from the equation of motion (\autoref{eq:eom}) when the mass matrix $\mathbf{M}$ and the acceleration vector $\pmb{\bm{\dot{\upsilon}}}$ are known. The hydrodynamic damping forces can be calculated by inserting the total force $\mathbf{F}$ from \autoref{eq:F_expanded} into \autoref{eq:eom} and then solve for $X_D$, $Y_D$, and $N_D$ as shown in \autoref{eq:ID_X}-\autoref{eq:ID_N}.
\begin{equation}
    \label{eq:ID_X}
    The forces acting on the ship during a maneuver can be estimated with inverse dynamics from the equation of motion (\autoref{eq:eom}) when the mass matrix $\mathbf{M}$ and the acceleration vector $\pmb{\bm{\dot{\upsilon}}}$ are known. The hydrodynamic damping forces can be calculated by inserting the total force $\mathbf{F}$ from \autoref{eq:F_expanded} into \autoref{eq:eom} and then solve for $X_D$, $Y_D$, and $N_D$ as shown in \autoref{eq:ID_X}-\autoref{eq:ID_N}.
\begin{equation}
    \label{eq:ID_X}
    \input{equations/methodology_ID.X_D}
\end{equation}
\begin{equation}
    \label{eq:ID_Y}
    \input{equations/methodology_ID.Y_D}
\end{equation}
\begin{equation}
    \label{eq:ID_N}
    \input{equations/methodology_ID.N_D}
\end{equation}
These expressions are used to estimate the forces acting on the ship during the FRMTs, which is then compared to the model force predictions. This is a more informative way to assess the model, than to compare its simulations with the FRMTs results.
\end{equation}
\begin{equation}
    \label{eq:ID_Y}
    The forces acting on the ship during a maneuver can be estimated with inverse dynamics from the equation of motion (\autoref{eq:eom}) when the mass matrix $\mathbf{M}$ and the acceleration vector $\pmb{\bm{\dot{\upsilon}}}$ are known. The hydrodynamic damping forces can be calculated by inserting the total force $\mathbf{F}$ from \autoref{eq:F_expanded} into \autoref{eq:eom} and then solve for $X_D$, $Y_D$, and $N_D$ as shown in \autoref{eq:ID_X}-\autoref{eq:ID_N}.
\begin{equation}
    \label{eq:ID_X}
    \input{equations/methodology_ID.X_D}
\end{equation}
\begin{equation}
    \label{eq:ID_Y}
    \input{equations/methodology_ID.Y_D}
\end{equation}
\begin{equation}
    \label{eq:ID_N}
    \input{equations/methodology_ID.N_D}
\end{equation}
These expressions are used to estimate the forces acting on the ship during the FRMTs, which is then compared to the model force predictions. This is a more informative way to assess the model, than to compare its simulations with the FRMTs results.
\end{equation}
\begin{equation}
    \label{eq:ID_N}
    The forces acting on the ship during a maneuver can be estimated with inverse dynamics from the equation of motion (\autoref{eq:eom}) when the mass matrix $\mathbf{M}$ and the acceleration vector $\pmb{\bm{\dot{\upsilon}}}$ are known. The hydrodynamic damping forces can be calculated by inserting the total force $\mathbf{F}$ from \autoref{eq:F_expanded} into \autoref{eq:eom} and then solve for $X_D$, $Y_D$, and $N_D$ as shown in \autoref{eq:ID_X}-\autoref{eq:ID_N}.
\begin{equation}
    \label{eq:ID_X}
    \input{equations/methodology_ID.X_D}
\end{equation}
\begin{equation}
    \label{eq:ID_Y}
    \input{equations/methodology_ID.Y_D}
\end{equation}
\begin{equation}
    \label{eq:ID_N}
    \input{equations/methodology_ID.N_D}
\end{equation}
These expressions are used to estimate the forces acting on the ship during the FRMTs, which is then compared to the model force predictions. This is a more informative way to assess the model, than to compare its simulations with the FRMTs results.
\end{equation}
These expressions are used to estimate the forces acting on the ship during the FRMTs, which is then compared to the model force predictions. This is a more informative way to assess the model, than to compare its simulations with the FRMTs results.
\end{equation}
\begin{equation}
    \label{eq:ID_N}
    The forces acting on the ship during a maneuver can be estimated with inverse dynamics from the equation of motion (\autoref{eq:eom}) when the mass matrix $\mathbf{M}$ and the acceleration vector $\pmb{\bm{\dot{\upsilon}}}$ are known. The hydrodynamic damping forces can be calculated by inserting the total force $\mathbf{F}$ from \autoref{eq:F_expanded} into \autoref{eq:eom} and then solve for $X_D$, $Y_D$, and $N_D$ as shown in \autoref{eq:ID_X}-\autoref{eq:ID_N}.
\begin{equation}
    \label{eq:ID_X}
    The forces acting on the ship during a maneuver can be estimated with inverse dynamics from the equation of motion (\autoref{eq:eom}) when the mass matrix $\mathbf{M}$ and the acceleration vector $\pmb{\bm{\dot{\upsilon}}}$ are known. The hydrodynamic damping forces can be calculated by inserting the total force $\mathbf{F}$ from \autoref{eq:F_expanded} into \autoref{eq:eom} and then solve for $X_D$, $Y_D$, and $N_D$ as shown in \autoref{eq:ID_X}-\autoref{eq:ID_N}.
\begin{equation}
    \label{eq:ID_X}
    \input{equations/methodology_ID.X_D}
\end{equation}
\begin{equation}
    \label{eq:ID_Y}
    \input{equations/methodology_ID.Y_D}
\end{equation}
\begin{equation}
    \label{eq:ID_N}
    \input{equations/methodology_ID.N_D}
\end{equation}
These expressions are used to estimate the forces acting on the ship during the FRMTs, which is then compared to the model force predictions. This is a more informative way to assess the model, than to compare its simulations with the FRMTs results.
\end{equation}
\begin{equation}
    \label{eq:ID_Y}
    The forces acting on the ship during a maneuver can be estimated with inverse dynamics from the equation of motion (\autoref{eq:eom}) when the mass matrix $\mathbf{M}$ and the acceleration vector $\pmb{\bm{\dot{\upsilon}}}$ are known. The hydrodynamic damping forces can be calculated by inserting the total force $\mathbf{F}$ from \autoref{eq:F_expanded} into \autoref{eq:eom} and then solve for $X_D$, $Y_D$, and $N_D$ as shown in \autoref{eq:ID_X}-\autoref{eq:ID_N}.
\begin{equation}
    \label{eq:ID_X}
    \input{equations/methodology_ID.X_D}
\end{equation}
\begin{equation}
    \label{eq:ID_Y}
    \input{equations/methodology_ID.Y_D}
\end{equation}
\begin{equation}
    \label{eq:ID_N}
    \input{equations/methodology_ID.N_D}
\end{equation}
These expressions are used to estimate the forces acting on the ship during the FRMTs, which is then compared to the model force predictions. This is a more informative way to assess the model, than to compare its simulations with the FRMTs results.
\end{equation}
\begin{equation}
    \label{eq:ID_N}
    The forces acting on the ship during a maneuver can be estimated with inverse dynamics from the equation of motion (\autoref{eq:eom}) when the mass matrix $\mathbf{M}$ and the acceleration vector $\pmb{\bm{\dot{\upsilon}}}$ are known. The hydrodynamic damping forces can be calculated by inserting the total force $\mathbf{F}$ from \autoref{eq:F_expanded} into \autoref{eq:eom} and then solve for $X_D$, $Y_D$, and $N_D$ as shown in \autoref{eq:ID_X}-\autoref{eq:ID_N}.
\begin{equation}
    \label{eq:ID_X}
    \input{equations/methodology_ID.X_D}
\end{equation}
\begin{equation}
    \label{eq:ID_Y}
    \input{equations/methodology_ID.Y_D}
\end{equation}
\begin{equation}
    \label{eq:ID_N}
    \input{equations/methodology_ID.N_D}
\end{equation}
These expressions are used to estimate the forces acting on the ship during the FRMTs, which is then compared to the model force predictions. This is a more informative way to assess the model, than to compare its simulations with the FRMTs results.
\end{equation}
These expressions are used to estimate the forces acting on the ship during the FRMTs, which is then compared to the model force predictions. This is a more informative way to assess the model, than to compare its simulations with the FRMTs results.
\end{equation}
These expressions are used to estimate the forces acting on the ship during the FRMTs, which is then compared to the model force predictions. This is a more informative way to assess the model, than to compare its simulations with the FRMTs results.
\end{equation}
\begin{equation}
    \label{eq:ID_Y}
    The forces acting on the ship during a maneuver can be estimated with inverse dynamics from the equation of motion (\autoref{eq:eom}) when the mass matrix $\mathbf{M}$ and the acceleration vector $\pmb{\bm{\dot{\upsilon}}}$ are known. The hydrodynamic damping forces can be calculated by inserting the total force $\mathbf{F}$ from \autoref{eq:F_expanded} into \autoref{eq:eom} and then solve for $X_D$, $Y_D$, and $N_D$ as shown in \autoref{eq:ID_X}-\autoref{eq:ID_N}.
\begin{equation}
    \label{eq:ID_X}
    The forces acting on the ship during a maneuver can be estimated with inverse dynamics from the equation of motion (\autoref{eq:eom}) when the mass matrix $\mathbf{M}$ and the acceleration vector $\pmb{\bm{\dot{\upsilon}}}$ are known. The hydrodynamic damping forces can be calculated by inserting the total force $\mathbf{F}$ from \autoref{eq:F_expanded} into \autoref{eq:eom} and then solve for $X_D$, $Y_D$, and $N_D$ as shown in \autoref{eq:ID_X}-\autoref{eq:ID_N}.
\begin{equation}
    \label{eq:ID_X}
    The forces acting on the ship during a maneuver can be estimated with inverse dynamics from the equation of motion (\autoref{eq:eom}) when the mass matrix $\mathbf{M}$ and the acceleration vector $\pmb{\bm{\dot{\upsilon}}}$ are known. The hydrodynamic damping forces can be calculated by inserting the total force $\mathbf{F}$ from \autoref{eq:F_expanded} into \autoref{eq:eom} and then solve for $X_D$, $Y_D$, and $N_D$ as shown in \autoref{eq:ID_X}-\autoref{eq:ID_N}.
\begin{equation}
    \label{eq:ID_X}
    \input{equations/methodology_ID.X_D}
\end{equation}
\begin{equation}
    \label{eq:ID_Y}
    \input{equations/methodology_ID.Y_D}
\end{equation}
\begin{equation}
    \label{eq:ID_N}
    \input{equations/methodology_ID.N_D}
\end{equation}
These expressions are used to estimate the forces acting on the ship during the FRMTs, which is then compared to the model force predictions. This is a more informative way to assess the model, than to compare its simulations with the FRMTs results.
\end{equation}
\begin{equation}
    \label{eq:ID_Y}
    The forces acting on the ship during a maneuver can be estimated with inverse dynamics from the equation of motion (\autoref{eq:eom}) when the mass matrix $\mathbf{M}$ and the acceleration vector $\pmb{\bm{\dot{\upsilon}}}$ are known. The hydrodynamic damping forces can be calculated by inserting the total force $\mathbf{F}$ from \autoref{eq:F_expanded} into \autoref{eq:eom} and then solve for $X_D$, $Y_D$, and $N_D$ as shown in \autoref{eq:ID_X}-\autoref{eq:ID_N}.
\begin{equation}
    \label{eq:ID_X}
    \input{equations/methodology_ID.X_D}
\end{equation}
\begin{equation}
    \label{eq:ID_Y}
    \input{equations/methodology_ID.Y_D}
\end{equation}
\begin{equation}
    \label{eq:ID_N}
    \input{equations/methodology_ID.N_D}
\end{equation}
These expressions are used to estimate the forces acting on the ship during the FRMTs, which is then compared to the model force predictions. This is a more informative way to assess the model, than to compare its simulations with the FRMTs results.
\end{equation}
\begin{equation}
    \label{eq:ID_N}
    The forces acting on the ship during a maneuver can be estimated with inverse dynamics from the equation of motion (\autoref{eq:eom}) when the mass matrix $\mathbf{M}$ and the acceleration vector $\pmb{\bm{\dot{\upsilon}}}$ are known. The hydrodynamic damping forces can be calculated by inserting the total force $\mathbf{F}$ from \autoref{eq:F_expanded} into \autoref{eq:eom} and then solve for $X_D$, $Y_D$, and $N_D$ as shown in \autoref{eq:ID_X}-\autoref{eq:ID_N}.
\begin{equation}
    \label{eq:ID_X}
    \input{equations/methodology_ID.X_D}
\end{equation}
\begin{equation}
    \label{eq:ID_Y}
    \input{equations/methodology_ID.Y_D}
\end{equation}
\begin{equation}
    \label{eq:ID_N}
    \input{equations/methodology_ID.N_D}
\end{equation}
These expressions are used to estimate the forces acting on the ship during the FRMTs, which is then compared to the model force predictions. This is a more informative way to assess the model, than to compare its simulations with the FRMTs results.
\end{equation}
These expressions are used to estimate the forces acting on the ship during the FRMTs, which is then compared to the model force predictions. This is a more informative way to assess the model, than to compare its simulations with the FRMTs results.
\end{equation}
\begin{equation}
    \label{eq:ID_Y}
    The forces acting on the ship during a maneuver can be estimated with inverse dynamics from the equation of motion (\autoref{eq:eom}) when the mass matrix $\mathbf{M}$ and the acceleration vector $\pmb{\bm{\dot{\upsilon}}}$ are known. The hydrodynamic damping forces can be calculated by inserting the total force $\mathbf{F}$ from \autoref{eq:F_expanded} into \autoref{eq:eom} and then solve for $X_D$, $Y_D$, and $N_D$ as shown in \autoref{eq:ID_X}-\autoref{eq:ID_N}.
\begin{equation}
    \label{eq:ID_X}
    The forces acting on the ship during a maneuver can be estimated with inverse dynamics from the equation of motion (\autoref{eq:eom}) when the mass matrix $\mathbf{M}$ and the acceleration vector $\pmb{\bm{\dot{\upsilon}}}$ are known. The hydrodynamic damping forces can be calculated by inserting the total force $\mathbf{F}$ from \autoref{eq:F_expanded} into \autoref{eq:eom} and then solve for $X_D$, $Y_D$, and $N_D$ as shown in \autoref{eq:ID_X}-\autoref{eq:ID_N}.
\begin{equation}
    \label{eq:ID_X}
    \input{equations/methodology_ID.X_D}
\end{equation}
\begin{equation}
    \label{eq:ID_Y}
    \input{equations/methodology_ID.Y_D}
\end{equation}
\begin{equation}
    \label{eq:ID_N}
    \input{equations/methodology_ID.N_D}
\end{equation}
These expressions are used to estimate the forces acting on the ship during the FRMTs, which is then compared to the model force predictions. This is a more informative way to assess the model, than to compare its simulations with the FRMTs results.
\end{equation}
\begin{equation}
    \label{eq:ID_Y}
    The forces acting on the ship during a maneuver can be estimated with inverse dynamics from the equation of motion (\autoref{eq:eom}) when the mass matrix $\mathbf{M}$ and the acceleration vector $\pmb{\bm{\dot{\upsilon}}}$ are known. The hydrodynamic damping forces can be calculated by inserting the total force $\mathbf{F}$ from \autoref{eq:F_expanded} into \autoref{eq:eom} and then solve for $X_D$, $Y_D$, and $N_D$ as shown in \autoref{eq:ID_X}-\autoref{eq:ID_N}.
\begin{equation}
    \label{eq:ID_X}
    \input{equations/methodology_ID.X_D}
\end{equation}
\begin{equation}
    \label{eq:ID_Y}
    \input{equations/methodology_ID.Y_D}
\end{equation}
\begin{equation}
    \label{eq:ID_N}
    \input{equations/methodology_ID.N_D}
\end{equation}
These expressions are used to estimate the forces acting on the ship during the FRMTs, which is then compared to the model force predictions. This is a more informative way to assess the model, than to compare its simulations with the FRMTs results.
\end{equation}
\begin{equation}
    \label{eq:ID_N}
    The forces acting on the ship during a maneuver can be estimated with inverse dynamics from the equation of motion (\autoref{eq:eom}) when the mass matrix $\mathbf{M}$ and the acceleration vector $\pmb{\bm{\dot{\upsilon}}}$ are known. The hydrodynamic damping forces can be calculated by inserting the total force $\mathbf{F}$ from \autoref{eq:F_expanded} into \autoref{eq:eom} and then solve for $X_D$, $Y_D$, and $N_D$ as shown in \autoref{eq:ID_X}-\autoref{eq:ID_N}.
\begin{equation}
    \label{eq:ID_X}
    \input{equations/methodology_ID.X_D}
\end{equation}
\begin{equation}
    \label{eq:ID_Y}
    \input{equations/methodology_ID.Y_D}
\end{equation}
\begin{equation}
    \label{eq:ID_N}
    \input{equations/methodology_ID.N_D}
\end{equation}
These expressions are used to estimate the forces acting on the ship during the FRMTs, which is then compared to the model force predictions. This is a more informative way to assess the model, than to compare its simulations with the FRMTs results.
\end{equation}
These expressions are used to estimate the forces acting on the ship during the FRMTs, which is then compared to the model force predictions. This is a more informative way to assess the model, than to compare its simulations with the FRMTs results.
\end{equation}
\begin{equation}
    \label{eq:ID_N}
    The forces acting on the ship during a maneuver can be estimated with inverse dynamics from the equation of motion (\autoref{eq:eom}) when the mass matrix $\mathbf{M}$ and the acceleration vector $\pmb{\bm{\dot{\upsilon}}}$ are known. The hydrodynamic damping forces can be calculated by inserting the total force $\mathbf{F}$ from \autoref{eq:F_expanded} into \autoref{eq:eom} and then solve for $X_D$, $Y_D$, and $N_D$ as shown in \autoref{eq:ID_X}-\autoref{eq:ID_N}.
\begin{equation}
    \label{eq:ID_X}
    The forces acting on the ship during a maneuver can be estimated with inverse dynamics from the equation of motion (\autoref{eq:eom}) when the mass matrix $\mathbf{M}$ and the acceleration vector $\pmb{\bm{\dot{\upsilon}}}$ are known. The hydrodynamic damping forces can be calculated by inserting the total force $\mathbf{F}$ from \autoref{eq:F_expanded} into \autoref{eq:eom} and then solve for $X_D$, $Y_D$, and $N_D$ as shown in \autoref{eq:ID_X}-\autoref{eq:ID_N}.
\begin{equation}
    \label{eq:ID_X}
    \input{equations/methodology_ID.X_D}
\end{equation}
\begin{equation}
    \label{eq:ID_Y}
    \input{equations/methodology_ID.Y_D}
\end{equation}
\begin{equation}
    \label{eq:ID_N}
    \input{equations/methodology_ID.N_D}
\end{equation}
These expressions are used to estimate the forces acting on the ship during the FRMTs, which is then compared to the model force predictions. This is a more informative way to assess the model, than to compare its simulations with the FRMTs results.
\end{equation}
\begin{equation}
    \label{eq:ID_Y}
    The forces acting on the ship during a maneuver can be estimated with inverse dynamics from the equation of motion (\autoref{eq:eom}) when the mass matrix $\mathbf{M}$ and the acceleration vector $\pmb{\bm{\dot{\upsilon}}}$ are known. The hydrodynamic damping forces can be calculated by inserting the total force $\mathbf{F}$ from \autoref{eq:F_expanded} into \autoref{eq:eom} and then solve for $X_D$, $Y_D$, and $N_D$ as shown in \autoref{eq:ID_X}-\autoref{eq:ID_N}.
\begin{equation}
    \label{eq:ID_X}
    \input{equations/methodology_ID.X_D}
\end{equation}
\begin{equation}
    \label{eq:ID_Y}
    \input{equations/methodology_ID.Y_D}
\end{equation}
\begin{equation}
    \label{eq:ID_N}
    \input{equations/methodology_ID.N_D}
\end{equation}
These expressions are used to estimate the forces acting on the ship during the FRMTs, which is then compared to the model force predictions. This is a more informative way to assess the model, than to compare its simulations with the FRMTs results.
\end{equation}
\begin{equation}
    \label{eq:ID_N}
    The forces acting on the ship during a maneuver can be estimated with inverse dynamics from the equation of motion (\autoref{eq:eom}) when the mass matrix $\mathbf{M}$ and the acceleration vector $\pmb{\bm{\dot{\upsilon}}}$ are known. The hydrodynamic damping forces can be calculated by inserting the total force $\mathbf{F}$ from \autoref{eq:F_expanded} into \autoref{eq:eom} and then solve for $X_D$, $Y_D$, and $N_D$ as shown in \autoref{eq:ID_X}-\autoref{eq:ID_N}.
\begin{equation}
    \label{eq:ID_X}
    \input{equations/methodology_ID.X_D}
\end{equation}
\begin{equation}
    \label{eq:ID_Y}
    \input{equations/methodology_ID.Y_D}
\end{equation}
\begin{equation}
    \label{eq:ID_N}
    \input{equations/methodology_ID.N_D}
\end{equation}
These expressions are used to estimate the forces acting on the ship during the FRMTs, which is then compared to the model force predictions. This is a more informative way to assess the model, than to compare its simulations with the FRMTs results.
\end{equation}
These expressions are used to estimate the forces acting on the ship during the FRMTs, which is then compared to the model force predictions. This is a more informative way to assess the model, than to compare its simulations with the FRMTs results.
\end{equation}
These expressions are used to estimate the forces acting on the ship during the FRMTs, which is then compared to the model force predictions. This is a more informative way to assess the model, than to compare its simulations with the FRMTs results.
\end{equation}
\begin{equation}
    \label{eq:ID_N}
    The forces acting on the ship during a maneuver can be estimated with inverse dynamics from the equation of motion (\autoref{eq:eom}) when the mass matrix $\mathbf{M}$ and the acceleration vector $\pmb{\bm{\dot{\upsilon}}}$ are known. The hydrodynamic damping forces can be calculated by inserting the total force $\mathbf{F}$ from \autoref{eq:F_expanded} into \autoref{eq:eom} and then solve for $X_D$, $Y_D$, and $N_D$ as shown in \autoref{eq:ID_X}-\autoref{eq:ID_N}.
\begin{equation}
    \label{eq:ID_X}
    The forces acting on the ship during a maneuver can be estimated with inverse dynamics from the equation of motion (\autoref{eq:eom}) when the mass matrix $\mathbf{M}$ and the acceleration vector $\pmb{\bm{\dot{\upsilon}}}$ are known. The hydrodynamic damping forces can be calculated by inserting the total force $\mathbf{F}$ from \autoref{eq:F_expanded} into \autoref{eq:eom} and then solve for $X_D$, $Y_D$, and $N_D$ as shown in \autoref{eq:ID_X}-\autoref{eq:ID_N}.
\begin{equation}
    \label{eq:ID_X}
    The forces acting on the ship during a maneuver can be estimated with inverse dynamics from the equation of motion (\autoref{eq:eom}) when the mass matrix $\mathbf{M}$ and the acceleration vector $\pmb{\bm{\dot{\upsilon}}}$ are known. The hydrodynamic damping forces can be calculated by inserting the total force $\mathbf{F}$ from \autoref{eq:F_expanded} into \autoref{eq:eom} and then solve for $X_D$, $Y_D$, and $N_D$ as shown in \autoref{eq:ID_X}-\autoref{eq:ID_N}.
\begin{equation}
    \label{eq:ID_X}
    \input{equations/methodology_ID.X_D}
\end{equation}
\begin{equation}
    \label{eq:ID_Y}
    \input{equations/methodology_ID.Y_D}
\end{equation}
\begin{equation}
    \label{eq:ID_N}
    \input{equations/methodology_ID.N_D}
\end{equation}
These expressions are used to estimate the forces acting on the ship during the FRMTs, which is then compared to the model force predictions. This is a more informative way to assess the model, than to compare its simulations with the FRMTs results.
\end{equation}
\begin{equation}
    \label{eq:ID_Y}
    The forces acting on the ship during a maneuver can be estimated with inverse dynamics from the equation of motion (\autoref{eq:eom}) when the mass matrix $\mathbf{M}$ and the acceleration vector $\pmb{\bm{\dot{\upsilon}}}$ are known. The hydrodynamic damping forces can be calculated by inserting the total force $\mathbf{F}$ from \autoref{eq:F_expanded} into \autoref{eq:eom} and then solve for $X_D$, $Y_D$, and $N_D$ as shown in \autoref{eq:ID_X}-\autoref{eq:ID_N}.
\begin{equation}
    \label{eq:ID_X}
    \input{equations/methodology_ID.X_D}
\end{equation}
\begin{equation}
    \label{eq:ID_Y}
    \input{equations/methodology_ID.Y_D}
\end{equation}
\begin{equation}
    \label{eq:ID_N}
    \input{equations/methodology_ID.N_D}
\end{equation}
These expressions are used to estimate the forces acting on the ship during the FRMTs, which is then compared to the model force predictions. This is a more informative way to assess the model, than to compare its simulations with the FRMTs results.
\end{equation}
\begin{equation}
    \label{eq:ID_N}
    The forces acting on the ship during a maneuver can be estimated with inverse dynamics from the equation of motion (\autoref{eq:eom}) when the mass matrix $\mathbf{M}$ and the acceleration vector $\pmb{\bm{\dot{\upsilon}}}$ are known. The hydrodynamic damping forces can be calculated by inserting the total force $\mathbf{F}$ from \autoref{eq:F_expanded} into \autoref{eq:eom} and then solve for $X_D$, $Y_D$, and $N_D$ as shown in \autoref{eq:ID_X}-\autoref{eq:ID_N}.
\begin{equation}
    \label{eq:ID_X}
    \input{equations/methodology_ID.X_D}
\end{equation}
\begin{equation}
    \label{eq:ID_Y}
    \input{equations/methodology_ID.Y_D}
\end{equation}
\begin{equation}
    \label{eq:ID_N}
    \input{equations/methodology_ID.N_D}
\end{equation}
These expressions are used to estimate the forces acting on the ship during the FRMTs, which is then compared to the model force predictions. This is a more informative way to assess the model, than to compare its simulations with the FRMTs results.
\end{equation}
These expressions are used to estimate the forces acting on the ship during the FRMTs, which is then compared to the model force predictions. This is a more informative way to assess the model, than to compare its simulations with the FRMTs results.
\end{equation}
\begin{equation}
    \label{eq:ID_Y}
    The forces acting on the ship during a maneuver can be estimated with inverse dynamics from the equation of motion (\autoref{eq:eom}) when the mass matrix $\mathbf{M}$ and the acceleration vector $\pmb{\bm{\dot{\upsilon}}}$ are known. The hydrodynamic damping forces can be calculated by inserting the total force $\mathbf{F}$ from \autoref{eq:F_expanded} into \autoref{eq:eom} and then solve for $X_D$, $Y_D$, and $N_D$ as shown in \autoref{eq:ID_X}-\autoref{eq:ID_N}.
\begin{equation}
    \label{eq:ID_X}
    The forces acting on the ship during a maneuver can be estimated with inverse dynamics from the equation of motion (\autoref{eq:eom}) when the mass matrix $\mathbf{M}$ and the acceleration vector $\pmb{\bm{\dot{\upsilon}}}$ are known. The hydrodynamic damping forces can be calculated by inserting the total force $\mathbf{F}$ from \autoref{eq:F_expanded} into \autoref{eq:eom} and then solve for $X_D$, $Y_D$, and $N_D$ as shown in \autoref{eq:ID_X}-\autoref{eq:ID_N}.
\begin{equation}
    \label{eq:ID_X}
    \input{equations/methodology_ID.X_D}
\end{equation}
\begin{equation}
    \label{eq:ID_Y}
    \input{equations/methodology_ID.Y_D}
\end{equation}
\begin{equation}
    \label{eq:ID_N}
    \input{equations/methodology_ID.N_D}
\end{equation}
These expressions are used to estimate the forces acting on the ship during the FRMTs, which is then compared to the model force predictions. This is a more informative way to assess the model, than to compare its simulations with the FRMTs results.
\end{equation}
\begin{equation}
    \label{eq:ID_Y}
    The forces acting on the ship during a maneuver can be estimated with inverse dynamics from the equation of motion (\autoref{eq:eom}) when the mass matrix $\mathbf{M}$ and the acceleration vector $\pmb{\bm{\dot{\upsilon}}}$ are known. The hydrodynamic damping forces can be calculated by inserting the total force $\mathbf{F}$ from \autoref{eq:F_expanded} into \autoref{eq:eom} and then solve for $X_D$, $Y_D$, and $N_D$ as shown in \autoref{eq:ID_X}-\autoref{eq:ID_N}.
\begin{equation}
    \label{eq:ID_X}
    \input{equations/methodology_ID.X_D}
\end{equation}
\begin{equation}
    \label{eq:ID_Y}
    \input{equations/methodology_ID.Y_D}
\end{equation}
\begin{equation}
    \label{eq:ID_N}
    \input{equations/methodology_ID.N_D}
\end{equation}
These expressions are used to estimate the forces acting on the ship during the FRMTs, which is then compared to the model force predictions. This is a more informative way to assess the model, than to compare its simulations with the FRMTs results.
\end{equation}
\begin{equation}
    \label{eq:ID_N}
    The forces acting on the ship during a maneuver can be estimated with inverse dynamics from the equation of motion (\autoref{eq:eom}) when the mass matrix $\mathbf{M}$ and the acceleration vector $\pmb{\bm{\dot{\upsilon}}}$ are known. The hydrodynamic damping forces can be calculated by inserting the total force $\mathbf{F}$ from \autoref{eq:F_expanded} into \autoref{eq:eom} and then solve for $X_D$, $Y_D$, and $N_D$ as shown in \autoref{eq:ID_X}-\autoref{eq:ID_N}.
\begin{equation}
    \label{eq:ID_X}
    \input{equations/methodology_ID.X_D}
\end{equation}
\begin{equation}
    \label{eq:ID_Y}
    \input{equations/methodology_ID.Y_D}
\end{equation}
\begin{equation}
    \label{eq:ID_N}
    \input{equations/methodology_ID.N_D}
\end{equation}
These expressions are used to estimate the forces acting on the ship during the FRMTs, which is then compared to the model force predictions. This is a more informative way to assess the model, than to compare its simulations with the FRMTs results.
\end{equation}
These expressions are used to estimate the forces acting on the ship during the FRMTs, which is then compared to the model force predictions. This is a more informative way to assess the model, than to compare its simulations with the FRMTs results.
\end{equation}
\begin{equation}
    \label{eq:ID_N}
    The forces acting on the ship during a maneuver can be estimated with inverse dynamics from the equation of motion (\autoref{eq:eom}) when the mass matrix $\mathbf{M}$ and the acceleration vector $\pmb{\bm{\dot{\upsilon}}}$ are known. The hydrodynamic damping forces can be calculated by inserting the total force $\mathbf{F}$ from \autoref{eq:F_expanded} into \autoref{eq:eom} and then solve for $X_D$, $Y_D$, and $N_D$ as shown in \autoref{eq:ID_X}-\autoref{eq:ID_N}.
\begin{equation}
    \label{eq:ID_X}
    The forces acting on the ship during a maneuver can be estimated with inverse dynamics from the equation of motion (\autoref{eq:eom}) when the mass matrix $\mathbf{M}$ and the acceleration vector $\pmb{\bm{\dot{\upsilon}}}$ are known. The hydrodynamic damping forces can be calculated by inserting the total force $\mathbf{F}$ from \autoref{eq:F_expanded} into \autoref{eq:eom} and then solve for $X_D$, $Y_D$, and $N_D$ as shown in \autoref{eq:ID_X}-\autoref{eq:ID_N}.
\begin{equation}
    \label{eq:ID_X}
    \input{equations/methodology_ID.X_D}
\end{equation}
\begin{equation}
    \label{eq:ID_Y}
    \input{equations/methodology_ID.Y_D}
\end{equation}
\begin{equation}
    \label{eq:ID_N}
    \input{equations/methodology_ID.N_D}
\end{equation}
These expressions are used to estimate the forces acting on the ship during the FRMTs, which is then compared to the model force predictions. This is a more informative way to assess the model, than to compare its simulations with the FRMTs results.
\end{equation}
\begin{equation}
    \label{eq:ID_Y}
    The forces acting on the ship during a maneuver can be estimated with inverse dynamics from the equation of motion (\autoref{eq:eom}) when the mass matrix $\mathbf{M}$ and the acceleration vector $\pmb{\bm{\dot{\upsilon}}}$ are known. The hydrodynamic damping forces can be calculated by inserting the total force $\mathbf{F}$ from \autoref{eq:F_expanded} into \autoref{eq:eom} and then solve for $X_D$, $Y_D$, and $N_D$ as shown in \autoref{eq:ID_X}-\autoref{eq:ID_N}.
\begin{equation}
    \label{eq:ID_X}
    \input{equations/methodology_ID.X_D}
\end{equation}
\begin{equation}
    \label{eq:ID_Y}
    \input{equations/methodology_ID.Y_D}
\end{equation}
\begin{equation}
    \label{eq:ID_N}
    \input{equations/methodology_ID.N_D}
\end{equation}
These expressions are used to estimate the forces acting on the ship during the FRMTs, which is then compared to the model force predictions. This is a more informative way to assess the model, than to compare its simulations with the FRMTs results.
\end{equation}
\begin{equation}
    \label{eq:ID_N}
    The forces acting on the ship during a maneuver can be estimated with inverse dynamics from the equation of motion (\autoref{eq:eom}) when the mass matrix $\mathbf{M}$ and the acceleration vector $\pmb{\bm{\dot{\upsilon}}}$ are known. The hydrodynamic damping forces can be calculated by inserting the total force $\mathbf{F}$ from \autoref{eq:F_expanded} into \autoref{eq:eom} and then solve for $X_D$, $Y_D$, and $N_D$ as shown in \autoref{eq:ID_X}-\autoref{eq:ID_N}.
\begin{equation}
    \label{eq:ID_X}
    \input{equations/methodology_ID.X_D}
\end{equation}
\begin{equation}
    \label{eq:ID_Y}
    \input{equations/methodology_ID.Y_D}
\end{equation}
\begin{equation}
    \label{eq:ID_N}
    \input{equations/methodology_ID.N_D}
\end{equation}
These expressions are used to estimate the forces acting on the ship during the FRMTs, which is then compared to the model force predictions. This is a more informative way to assess the model, than to compare its simulations with the FRMTs results.
\end{equation}
These expressions are used to estimate the forces acting on the ship during the FRMTs, which is then compared to the model force predictions. This is a more informative way to assess the model, than to compare its simulations with the FRMTs results.
\end{equation}
These expressions are used to estimate the forces acting on the ship during the FRMTs, which is then compared to the model force predictions. This is a more informative way to assess the model, than to compare its simulations with the FRMTs results.
\end{equation}
These expressions are used to estimate the forces acting on the ship during the FRMTs, which is then compared to the model force predictions. This is a more informative way to assess the model, than to compare its simulations with the FRMTs results.