The forces acting on the ship during a maneuver can be estimated with inverse dynamics from the equation of motion (\autoref{eq:eom}) when the mass matrix $\mathbf{M}$ and the acceleration vector $\pmb{\bm{\dot{\upsilon}}}$ are known. The hydrodynamic damping forces can be calculated by inserting the total force $\mathbf{F}$ from \autoref{eq:F_expanded} into \autoref{eq:eom} and then solve for $X_D$, $Y_D$, and $N_D$ as shown in \autoref{eq:ID_X}-\autoref{eq:ID_N}.
\begin{equation}
    \label{eq:ID_X}
    X_{D} = - X_{\dot{u}} \dot{u} + Y_{\dot{r}} r^{2} + Y_{\dot{v}} r v + \dot{u} m - m r^{2} x_{G} - m r v
\end{equation}
\begin{equation}
    \label{eq:ID_Y}
    Y_{D} = - X_{\dot{u}} r u - Y_{\dot{r}} \dot{r} - Y_{\dot{v}} \dot{v} + \dot{r} m x_{G} + \dot{v} m + m r u
\end{equation}
\begin{equation}
    \label{eq:ID_N}
    N_{D} = I_{z} \dot{r} - N_{\dot{r}} \dot{r} - N_{\dot{v}} \dot{v} + X_{\dot{u}} u v - Y_{\dot{r}} r u - Y_{\dot{v}} u v + \dot{v} m x_{G} + m r u x_{G}
\end{equation}
These expressions are used to estimate the forces acting on the ship during the FRMTs, which is then compared to the model force predictions. This is a more informative way to assess the model, than to compare its simulations with the FRMTs results.