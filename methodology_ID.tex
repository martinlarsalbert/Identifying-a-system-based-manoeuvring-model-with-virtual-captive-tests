\noindent Instead of using the equation of motion (\autoref{eq:eom}) to predict the acceleration $\pmb{\bm{\dot{\upsilon}}}$, this equation can instead be inverted to describe the inverse dynamics of the ship as:
\begin{equation}
    \label{eq:ID_X}
    \left.\begin{aligned}
    X_{D} = - X_{\dot{u}} \dot{u} + Y_{\dot{r}} r^{2} + Y_{\dot{v}} r v + \dot{u} m - m r^{2} x_{G} - m r v \\
    Y_{D} = - X_{\dot{u}} r u - Y_{\dot{r}} \dot{r} - Y_{\dot{v}} \dot{v} + \dot{r} m x_{G} + \dot{v} m + m r u \\
    N_{D} = I_{z} \dot{r} - N_{\dot{r}} \dot{r} - N_{\dot{v}} \dot{v} + X_{\dot{u}} u v - Y_{\dot{r}} r u - Y_{\dot{v}} u v + \dot{v} m x_{G} + m r u x_{G}
    \end{aligned}\right\}
\end{equation}
% \begin{equation}
%     \label{eq:ID_Y}
%     Y_{D} = - X_{\dot{u}} r u - Y_{\dot{r}} \dot{r} - Y_{\dot{v}} \dot{v} + \dot{r} m x_{G} + \dot{v} m + m r u
% \end{equation}
% \begin{equation}
%     \label{eq:ID_N}
%     N_{D} = I_{z} \dot{r} - N_{\dot{r}} \dot{r} - N_{\dot{v}} \dot{v} + X_{\dot{u}} u v - Y_{\dot{r}} r u - Y_{\dot{v}} u v + \dot{v} m x_{G} + m r u x_{G}
% \end{equation}
It can be used to estimate the forces acting on the ship based on experimental data from the FRMTs. They can then be compared to the model force predictions. This is a more informative way to assess the model, than to compare its simulations with the FRMTs results.