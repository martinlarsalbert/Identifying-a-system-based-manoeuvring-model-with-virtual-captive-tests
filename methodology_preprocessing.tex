The roll degree of freedom is not included in the 3 DOF system based model. But the roll angle may influence the sway motion for ships with low GM, such as the wPCC. To minimize this effect, the ship motions are expressed at the ship's roll center. The roll center was determined from roll decay tests at the same speed as the manoeuvring tests were conducted. \autoref{fig:roll_center} shows how the sway motion is reduced by this transformation during the roll decay test (\autoref{fig:roll_center_rolldecay}) and during one of the zigzag tests (\autoref{fig:roll_center_zigzag}) for the wPCC. This transformation is not necessary for ships with higher GM, such as the Optiwise.
\begin{figure}[h]
     \centering
     \begin{subfigure}[b]{0.49\textwidth}
         \centering
         \includesvg{figures/result_roll_centre_wPCC.y0_at_roll_center.svg}
        \caption{Simulations wPCC Zigzag10/10 to port.}
        \label{fig:roll_center_rolldecay}
     \end{subfigure}
     \hfill
     \begin{subfigure}[b]{0.49\textwidth}
        \centering
        \includesvg{figures/result_roll_centre_wPCC.zz10_v1d_at_roll_centre}
        \caption{Simulations wPCC Zigzag20/20 to starboard.}
        \label{fig:roll_center_zigzag}
     \end{subfigure}
        \caption{Comparison between zigzag tests with wPCC from experiments and simulations with a model equipped with either a polynomial rudder or semi-empirical rudder model.}
        \label{fig:roll_center}
\end{figure}