\noindent The roll degree of freedom is not included in the 3 DOF manoeuvring model in this study. However, the roll angle may influence the sway motion for ships with low GM, such as the wPCC. To minimize this effect, the ship motions are expressed at the ship's roll center. The roll center was determined from roll decay tests at the same speed as the manoeuvring tests were conducted. \autoref{fig:roll_center} shows how the sway motion is reduced by this transformation during the roll decay test (\autoref{fig:roll_center_rolldecay}) and during one of the zigzag tests (\autoref{fig:roll_center_zigzag}) for the wPCC. This transformation is not necessary for ships with higher GM, such as the Optiwise.
\begin{figure}[h]
     \centering
     \begin{subfigure}[b]{0.49\textwidth}
         \centering
         \includesvg{figures/result_roll_centre_wPCC.y0_at_roll_center.svg}
        \caption{Roll decay test.}
        \label{fig:roll_center_rolldecay}
     \end{subfigure}
     \hfill
     \begin{subfigure}[b]{0.49\textwidth}
        \centering
        \includesvg{figures/result_roll_centre_wPCC.zz10_v1d_at_roll_centre}
        \caption{Zigzag test.}
        \label{fig:roll_center_zigzag}
     \end{subfigure}
        \caption{Sway motion transformed to roll center during tests with wPCC.}
        \label{fig:roll_center}
\end{figure}

The raw data from the experiments do not contain any measurements of velocity and acceleration during the tests, only the position and heading of the ship. The velocity and acceleration were therefore estimated by an extended Kalman filter (EKF), with the manoeuvring model as the predictor as described in \citet{alexanderssonSystemIdentificationVessel2022}.

