A ship is a system that transports passengers and goods on water. Manoeuvring, is a fundamental aspect of this system. The ability to control the ship from departure to destination along a desired route, avoiding obstacles such as land and other ships is an essential. The IMO standards for manoeuvrability highlights the importance of the ship manoeuvring. The performance can be assessed by conducting standard manoeuvres with the real built system – the ship, during sea trials. Revealing a substandard manoeuvring performance at this stage is however undesirable, where the possibilities for design changes are very limited and costly. This calls for earlier assessments, before the ship is build, which is traditionally done in scale model tests with a free sailing model.

With the advancement of numerical methods, CFD is an alternative, which may be more cost efficient and potentially more accurate, where the scale effect problem can be avoided by conducting full scale simulations. Direct CFD calculations in the time domain, visiting all the states during a manoeuvre, are however computationally expensive, even more expensive than the scale model tests and is therefore mostly applied in a research context, than in commercial ship building projects.

Virtual captive tests (VCT) can be used to reduce the computational effort by only calculating a few potential states of a manoeuvre with CFD. The calculated forces acting on the ship for the sampled states can be used to develop a prediction model, which can be used to simulate the unseen states during the manoeuvre.  

The CFD calculations for the VCT in this paper are assumed to have sufficient accuracy. The accuracy of the CFD itself will therefore not be questioned. Instead, the accuracy of the hydrodynamic discretization in developing a system based model from a set of VCT is the main focus.  

\citep{abkowitz_ship_1964}