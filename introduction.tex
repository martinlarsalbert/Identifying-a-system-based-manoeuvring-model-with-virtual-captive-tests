Ships greater than 100 meters must meet maneuverability requirements to ensure navigation safety \citep{imoStandardsShipManoeuvrability2002}. As the urgent demands to decarbonize the shipping industry increase, wind-assisted propulsion systems (WAPS) can become essential equipment on future ships \citep{nelissenStudyAnalysisMarket2016}. Ships with WAPS are often equipped with large rudders to compensate for large side forces. The maneuvering characteristics of the ship can be predicted by various approaches, such as model tests \citep{ittcManeuveringCommitteeITTC2008}, numerical simulations using computational fluid dynamics (CFD) \citep{elmoctarRANSBasedSimulatedShip2014,dumanTurnZigzagManoeuvres2022}, and mathematical modeling \citep{abkowitzMEASUREMENTHYDRODYNAMICCHARACTERISTICS1980,fossenHandbookMarineCraft2011,yasukawaIntroductionMMGStandard2015,alexanderssonSystemIdentificationPhysicsinformed2024b}. 

Free-running model tests are often recognized as the most reliable way to predict a ship's maneuverability \citep{ittcITTCRecommendedProcedures2008}. They often served as references for other prediction methods, for example for validation in several benchmark studies of different CFD simulations such as by \citet{sternExperienceSIMMAN20082011}, \citet{sakamotoURANSSimulationsStatic2012}, \citet{yoonBenchmarkCFDValidation2015a}, \citet{yasukawaValidation6DOFMotion2021}, and they are also used for system identification of mathematical maneuvering models such as \citet{luoParameterIdentificationShip2016}, \citet{xuUncertaintyAnalysisHydrodynamic2019}, \citet{wangOptimalDesignExcitation2020}, \citet{alexanderssonSystemIdentificationVessel2022}. Due to the flexibility of changing the ship hull geometry in a numerical analysis, CFD-based maneuvering simulations play an important role in studying the maneuverability of a ship. Similarly to model tests, CFD-based methods have been utilized to simulate all different maneuvering test scenarios. The most complex is to employ high-fidelity CFD analysis to simulate free-running tests with a steering rudder and a rotating propeller in the time domain \citep{dubbiosoTurningAbilityAnalysis2016a, islamEstimationHydrodynamicDerivatives2018}. These CFD-based methods were used to simulate very complicated navigation scenarios even without too many model test experiences, such as the free running ship motions in the following waves \citep{arakiImprovedManeuveringBasedMathematical2019}, the hull-propulsor-engine interaction during a ship’s maneuvering \citep{elmoctarRANSBasedSimulatedShip2014}, and the turn and zigzag maneuvers for catamarans \citep{dumanTurnZigzagManoeuvres2022}.

However, due to the huge cost in terms of either infrastructure or computational resources related to model tests and high-fidelity CFD methods, different mathematical maneuvering models and corresponding parameter identification methods have been extensively studied to predict a ship’s maneuverability. The so-called system identification methods for ship maneuvering are based on model tests or CFD simulations as input, but can help to significantly reduce the number of tests in these methods \citep{lokukalugep.pereraSystemIdentificationVessel2016,alexanderssonSystemIdentificationPhysicsinformed2024b}. When model or field tests were used for identification, a large effort should be made to find robust algorithms for data cleaning and data-driven models \citep{revestidoherreroTwostepIdentificationNonlinear2012,alexanderssonSystemIdentificationVessel2022,duShipManeuveringPrediction2022}; while when CFD-based simulations were used as input, the sensitivities of CFD analysis settings should be investigated since the data from the numerical simulations are ideal \citep{liuPredictionsShipManeuverability2018}.

For conventional ships, different mathematical models to describe their maneuverability have been well investigated, while empirical formulas for some maneuvering components are available \citep{yasukawaIntroductionMMGStandard2015}. For ships with WAPS installed, ship maneuverability can be significantly affected due to extra WAPS and large rudders, which can cause different flow characteristics for propeller-rudder interactions and obvious drifting. There is a lack of experience how their maneuverability should be modeled. In this study, a systematic approach is proposed to obtain hydrodynamic coefficients and hydrodynamic derivatives in an MMG-style model, as well as modeling of rudder forces under various drifting conditions. A limited amount of test data and cheap CFD-based virtual captive tests are proposed to obtain some necessary parameters in the MMG model. In this study, two ships designed with WAPS and different propulsion settings are used to validate the proposed approach. 

For a complete description of the proposed method, the remaining part of the paper is organized as follows. 
The maneuvering model is first introduced in \autoref{sec:model}. The two test cases are presented in \autoref{sec:test_cases} together with the method to preprocess the experimental data and how the inertia of the ships was determined. The proposed method for parameter identification is introduced in \autoref{sec:PIT}. The results are presented in \autoref{sec:results_wpcc} for wPCC and in \autoref{sec:results_optiwise} for Optiwise followed by the study's conclusions in \autoref{sec:conclusions}.


%A ship is a system that transports passengers and goods on water. Manoeuvring, is a fundamental aspect of this system. The ability to control the ship from departure to destination along a desired route, avoiding obstacles such as land and other ships is an essential. The IMO standards for manoeuvrability highlights the importance of the ship manoeuvring. The performance can be assessed by conducting standard manoeuvres with the real built system – the ship, during sea trials. Revealing a substandard manoeuvring performance at this stage is however undesirable, where the possibilities for design changes are very limited and costly. This calls for earlier assessments, before the ship is build, which is traditionally done in scale model tests with a free sailing model.

%With the advancement of numerical methods, CFD is an alternative, which may be more cost efficient and potentially more accurate, where the scale effect problem can be avoided by conducting full scale simulations. Direct CFD calculations in the time domain, visiting all the states during a manoeuvre, are however computationally expensive, even more expensive than the scale model tests and is therefore mostly applied in a research context, than in commercial ship building projects.

%Virtual captive tests (VCT) can be used to reduce the computational effort by only calculating a few potential states of a manoeuvre with CFD. The calculated forces acting on the ship for the sampled states can be used to develop a prediction model, which can be used to simulate the unseen states during the manoeuvre.  

%The CFD calculations for the VCT in this paper are assumed to have sufficient accuracy. The accuracy of the CFD itself will therefore not be questioned. Instead, the accuracy of the hydrodynamic discretization in developing a system based model from a set of VCT is the main focus.  