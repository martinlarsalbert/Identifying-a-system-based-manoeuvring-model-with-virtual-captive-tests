Ships greater than 100 meters must meet maneuverability requirements to ensure navigation safety \citep{imoStandardsShipManoeuvrability2002}. As the urgent demands to decarbonize the shipping industry increase, wind-assisted propulsion systems (WAPS) can become essential equipment on future ships \citep{nelissenStudyAnalysisMarket2016}. Ships with WAPS are often equipped with large rudders to compensate for significant side forces.

The maneuvering characteristics of a ship can be predicted using various approaches. The possibilities differ significantly between an existing ship, where operational data is available, and a new ship, where no operational data is yet available. The data can be categorized as either free running test (FT) data, where the ship is exposed to forces (the input) and resulting trajectories are recorded (the output), or the reversed scenario in captive test (CT) data, where the trajectory is the input and resulting forces are the output. FT data can be collected for existing ships, but also for new ships in free-running model tests, which are often recognized as the most reliable way to predict a ship's maneuverability \citep{ittcITTCRecommendedProcedures2008}. FT data can also be collected from direct CFD calculations at full scale, avoiding the scale effects of model tests. Regardless of the method used to obtain FT data, it only describes the maneuvering characteristics for a set of preset and often standardized maneuvers \citep{imoStandardsShipManoeuvrability2002} that do not generalize to other speeds or maneuvers, which can be a drawback of this approach. However, generalized results can be obtained with system identification, where a prediction model can be identified from the FT data as described by \citet{luoParameterIdentificationShip2016, xuUncertaintyAnalysisHydrodynamic2019, wangOptimalDesignExcitation2020, alexanderssonSystemIdentificationVessel2022, haoRecurrentNeuralNetworks2022a, kimValidation4DOFManeuvering2024, alexanderssonSystemIdentificationPhysicsinformed2024b}.

Most ship designs with WAPS are new building projects, where the ship does not yet exist, so the maneuvering characteristics must be studied by model tests or CFD calculations. Free running tests can be conducted with CFD \citep{sakamotoURANSSimulationsStatic2012, elmoctarRANSBasedSimulatedShip2014, dumanTurnZigzagManoeuvres2022}. However, if high accuracy is required, this might be a computationally expensive option in most cases. A more efficient approach is to conduct captive tests with CFD in virtual captive tests (VCT) as input to a system-based prediction model that can simulate the desired maneuvers as described by \citet{simonsenKCSPMMTests2014, elmoctarRANSBasedSimulatedShip2014, hajivandVirtualSimulationManeuvering2015, yoonBenchmarkCFDValidation2015c, liuPredictionsShipManeuverability2018}. These CFD-based methods were used to simulate very complicated navigation scenarios even without extensive model test experience, such as free-running ship movements in following waves \citep{Araki2019}, the hull-propulsor-engine interaction during maneuvering of a ship \citep{elmoctarRANSBasedSimulatedShip2014}, and turn and zigzag maneuvers for catamarans \citep{dumanTurnZigzagManoeuvres2022}.

For conventional ships, various mathematical models describing their maneuverability have been thoroughly investigated, with empirical formulas available for certain maneuvering components \citep{yasukawaIntroductionMMGStandard2015}. However, the maneuverability of ships equipped with WAPS can be significantly affected due to the additional WAPS and larger rudders. These modifications lead to altered flow characteristics in propeller-rudder interactions and noticeable drifting.
Several models have been proposed to address these changes \citep{violaNumericalMethodDesign2015,tillig4DOFSimulation2019,kjellbergSailingPerformanceWindPowered2023,guzelbulutInvestigationEfficiencyWindassisted2024a}. Despite this, many of these models remain unvalidated due to a lack of validation cases. This study aims to address this gap by using model tests with two ships designed with WAPS and different propulsion settings.
In this study, a systematic approach is proposed to obtain hydrodynamic coefficients and derivatives within an MMG-style model, alongside modeling rudder forces under various drifting conditions. Additionally, cost-effective CFD-based virtual captive tests are suggested to acquire some necessary parameters for the MMG model.

For a complete description of the proposed method, the remaining part of the paper is organized as follows. The maneuvering model is first introduced in \autoref{sec:model}. The two test cases are presented in \autoref{sec:test_cases} together with the method to preprocess the experimental data and how the inertia of the ships was determined. The proposed method for parameter identification is introduced in \autoref{sec:PIT}. The results are presented in \autoref{sec:results_wpcc} for wPCC and in \autoref{sec:results_optiwise} for Optiwise, followed by the study conclusions in \autoref{sec:conclusions}.


%A ship is a system that transports passengers and goods on water. Manoeuvring, is a fundamental aspect of this system. The ability to control the ship from departure to destination along a desired route, avoiding obstacles such as land and other ships is an essential. The IMO standards for manoeuvrability highlights the importance of the ship manoeuvring. The performance can be assessed by conducting standard manoeuvres with the real built system – the ship, during sea trials. Revealing a substandard manoeuvring performance at this stage is however undesirable, where the possibilities for design changes are very limited and costly. This calls for earlier assessments, before the ship is build, which is traditionally done in scale model tests with a free sailing model.

%With the advancement of numerical methods, CFD is an alternative, which may be more cost efficient and potentially more accurate, where the scale effect problem can be avoided by conducting full scale simulations. Direct CFD calculations in the time domain, visiting all the states during a manoeuvre, are however computationally expensive, even more expensive than the scale model tests and is therefore mostly applied in a research context, than in commercial ship building projects.

%Virtual captive tests (VCT) can be used to reduce the computational effort by only calculating a few potential states of a manoeuvre with CFD. The calculated forces acting on the ship for the sampled states can be used to develop a prediction model, which can be used to simulate the unseen states during the manoeuvre.  

%The CFD calculations for the VCT in this paper are assumed to have sufficient accuracy. The accuracy of the CFD itself will therefore not be questioned. Instead, the accuracy of the hydrodynamic discretization in developing a system based model from a set of VCT is the main focus.  