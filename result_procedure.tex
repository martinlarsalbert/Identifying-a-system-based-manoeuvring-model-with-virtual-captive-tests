\noindent Based on the above-proposed procedure, i.e., obtaining the hydrodynamic coefficients of such as added masses $\boldsymbol{M}$, total hydrodynamic forces $\boldsymbol{M}$ applied on the ship in Eq.(\ref{eq:F_expanded}) from VCTs, propeller force in Eq.(\ref{eq:X_P}) and rudder forces Eq.(\ref{eq:rudder}) from the MMG model, all the hydrodynamic derivative for the wPCC ship can be identified by the least square regression method based on the Eqs.(\ref{eq:XYN_H}, \ref{eq:XYN_H_prime}, \ref{eq:X_H_VCT}) for ship hull forces. The identified hydrodynamic derivatives are listed in Table \ref{tab:parameters}, where the hydrodynamic added mass terms are given in Table \ref{tab:added_masses}. For the wPCC ship, the semi-empirical rudder model proposed by \cite{alexanderssonSystemIdentificationPhysicsinformed2024b} is used to obtain the rudder force. The parameters used in the semi-empirical model are listed in Table \ref{tab:wPCC_other_parameters}. The validation of the proposed system identification method for the wPCC ship maneuvering is presented in detail as follows.