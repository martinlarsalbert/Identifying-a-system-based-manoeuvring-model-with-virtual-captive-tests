\noindent In the MMG model introduced in \citet{yasukawaIntroductionMMGStandard2015}, if the rudder tangential force is neglected, the effective rudder forces $X_R, Y_R$ and $N_R$ can be described as:
\begin{equation}
   \label{eq:rudder}
  \left.\begin{aligned}
  X_R & = - (1-t_R) F_N \sin \delta\\
  Y_R & = - (1+ \alpha_H) F_N \cos \delta\\
  N_R & = - (x_R + \alpha_H x_H) F_N \cos \delta
\end{aligned}\right\}
\end{equation}
where $F_N$ denotes the hydrodynamic forces applied on the normal direction of the rudder, $\delta$ denotes the rudder angle, and $t_R$, $\alpha_H$, $x_R, x_H$ denote the coefficients describing hydrodynamic interaction between rudder and hull, while $x_R, x_H$ represent longitudinal locations of the additional lateral forces applied on rudder and ship hull, respectively. The semi-empirical models for these coefficients are described in detail in \citet{yasukawaIntroductionMMGStandard2015} and \citet{alexanderssonSystemIdentificationPhysicsinformed2024b}. The key component in estimating effective rudder forces in Eq.(\ref{eq:rudder}) is to obtain the normal force applied on the rudder expressed by:
\begin{equation}
    \label{eq:rudder_normal}
    F_N = 1/2 \rho A_R U_R^{2} f_{\alpha} \sin \alpha_R
\end{equation}
where $A_R, f_\alpha$ denote the rudder area and the rudder lift gradient coefficient, respectively, and $\alpha_R$ is the angle of effective flow to the rudder, which has significant impact on the rudder force in Eq.(\ref{eq:rudder_normal}) due the asymmetric properties of flows caused by ship drifting.
In this study, a new quadratic rudder model is proposed with two enhancements to the original MMG rudder model as in \citet{yasukawaIntroductionMMGStandard2015}  in order to get a better fit to the VCT data. The first enhancement is to add the angle of initial inflow to rudder  $\gamma_0$ for the calculation of the angle of effective inflow acting to rudder $\alpha_R$:  
\begin{equation}
    \label{eq:alpha_R2}
    %\alpha_{R} = \delta + \gamma_{0} + \operatorname{atan}{\left(\frac{v_{R}}{u_{R}} \right)}
    \alpha_{R} = \delta + \underbrace{\gamma_{0}}_{\text{proposed}} + \operatorname{atan}{\left(\frac{v_{R}}{u_{R}} \right)}
\end{equation}
which allows the rudder model to produce a side force in the straight ahead condition, due to unsymmetrical flow from the propeller.

The other enhancement is to allow for a quadratic relationship between the flow straightening coefficient $\gamma_R$ and the effective inflow angle $\beta_R$ by introducing two new coefficients $\gamma_{R2neg}$, and $\gamma_{R2pos}$ as:  
\begin{equation}
    \label{eq:gamma_R2}
    \gamma_{R} = \begin{cases} \gamma_{R2 neg} \left|{\beta_{R}}\right| + \gamma_{R neg} & \text{for}\: \beta_{R} \leq 0 \\\gamma_{R2 pos} \left|{\beta_{R}}\right| + \gamma_{R pos} & \text{otherwise} \end{cases}
\end{equation}
which can be used to calculate the transverse velocity of flow passing the rudder as in Eq.(\ref{eq:alpha_R2}). 
