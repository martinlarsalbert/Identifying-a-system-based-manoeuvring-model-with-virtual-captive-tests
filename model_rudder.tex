The twin rudder wPCC simulation model was equipped with the semi-empirical rudder model, as described in \citet{alexanderssonSystemIdentificationPhysicsinformed2024b}.

The single rudder Optiwise simulation model was equipped with two versions of the MMG rudder model \citep{yasukawaIntroductionMMGStandard2015}.
A modified quadratic MMG rudder model is proposed in the present paper, with two enhancements to the original MMG rudder model. The first enhancement is to add the rudder initial inflow angle $\gamma_0$ to the calculation of the effective inflow angle to the rudder $\alpha_R$, by replacing equation 21 in \citet{yasukawaIntroductionMMGStandard2015} with the modified equation (\autoref{eq:alpha_R2}). This allows the rudder model to produce a side force in the straight ahead condition, due to unsymmetrical flow from the propeller. 
\begin{equation}
    \label{eq:alpha_R2}
    %\alpha_{R} = \delta + \gamma_{0} + \operatorname{atan}{\left(\frac{v_{R}}{u_{R}} \right)}
    \alpha_{R} = \delta + \underbrace{\gamma_{0}}_{~} + \operatorname{atan}{\left(\frac{v_{R}}{u_{R}} \right)}
\end{equation}
The other enhancement is to allow for a quadratic relationship between the flow straightening coefficient $\gamma_R$ and the effective inflow angle $\beta_R$ by introducing two new coefficients $\gamma_{R2neg}$, and $\gamma_{R2pos}$ as shown in \autoref{eq:gamma_R2}. The $\gamma_R$ calculated by this expression can be used in the equation 23 in \citet{yasukawaIntroductionMMGStandard2015} to calculate the rudder transverse velocity.  
\begin{equation}
    \label{eq:gamma_R2}
    \gamma_{R} = \begin{cases} \gamma_{R2 neg} \left|{\beta_{R}}\right| + \gamma_{R neg} & \text{for}\: \beta_{R} \leq 0 \\\gamma_{R2 pos} \left|{\beta_{R}}\right| + \gamma_{R pos} & \text{otherwise} \end{cases}
\end{equation}

