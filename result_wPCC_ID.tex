Forces predicted with the wPCC model equipped with either the semi-empirical rudder or the polynomial rudder have been compared with forces from the experiment, estimated by inverse dynamics. 
%10/10
Both rudder models have reasonably good agreement with the experimental results from the zigzag10/10, as shown in \autoref{fig:ID_wPCC_10}. 
%20/20
For the 20/20 test, as shown in \autoref{fig:ID_wPCC_20}, only the polynomial rudder model seems to capture the forces correctly. There are however some yawing moment $N_D$ deviations at t=9.5 s, and t=27 s, during the changes of rudder angle. 

Extra VCT calculations have been conducted for some of the states during the maneuver. There is generally a good agreement between these calculated states and the experimental forces. There are however small deviations for the yawing moment, especially around the rudder angle changes. 

The $N_D$ deviations between the experiments and the force prediction model with semi-empirical rudder are of the same magnitude as the the difference between the rudder yawing moment $N_R$ predictions with the polynomial and semi-empirical rudder. This is especially pronounced for the zigzag20/20 (\autoref{fig:ID_wPCC_20}) which suggests that the deviations of the total yawing moment $N_D$ from the semi-empirical model originates from the rudder model predictions of $N_R$. 
\begin{figure}[h]
     \centering
     \begin{subfigure}[b]{\textwidth}
         \centering
         \includesvg{figures/results_wPCC_ID.zigzag 10_10.svg}
        \caption{Forces wPCC Zigzag10/10 to port.}
        \label{fig:ID_wPCC_10}
     \end{subfigure}
     \vfill
     \begin{subfigure}[b]{\textwidth}
         \includesvg{figures/results_wPCC_ID.zigzag 20_20.svg}
        \caption{Forces wPCC Zigzag20/20 to starboard.}
        \label{fig:ID_wPCC_20}
     \end{subfigure}
        \caption{Comparison between forces during zigzag tests with wPCC estimated with inverse dynamics from the experiments and predictions with a model equipped with either a polynomial rudder or semi-empirical rudder model. Forces from VCT calculations of some interesting states have also been added.}
        \label{fig:ID_wPCC}
\end{figure}        