Similar comparisons are shown for the drift angle tests in \autoref{fig:drift_angle_optiwise} and the circle tests in \autoref{fig:circle_optiwise}. 
%Drift angle
\begin{figure}[h]
     \centering
     \begin{subfigure}[b]{0.32\textwidth}
         \centering
         \includesvg{figures/results_optiwise_VCT.drift_angle_X.svg}
        \caption{Surge force.}
        \label{fig:drift_angle_X_optiwise}
     \end{subfigure}
     \hfill
     \begin{subfigure}[b]{0.32\textwidth}
         \centering
         \includesvg{figures/results_optiwise_VCT.drift_angle_Y.svg}
        \caption{Sway force.}
        \label{fig:drift_angle_Y_optiwise}
     \end{subfigure}
     \hfill
     \begin{subfigure}[b]{0.32\textwidth}
         \centering
         \includesvg{figures/results_optiwise_VCT.drift_angle_N.svg}
        \caption{Yawing moment.}
        \label{fig:drift_angle_N_optiwise}
     \end{subfigure}
    \caption{Optiwise drift angle tests from VCT (dots) and predictions (lines).}
    \label{fig:drift_angle_optiwise}
\end{figure}
%Circle
\begin{figure}[h]
     \centering
     \begin{subfigure}[b]{0.32\textwidth}
         \centering
         \includesvg{figures/results_optiwise_VCT.circle_X.svg}
        \caption{Surge force.}
        \label{fig:drift_angle_X_optiwise}
     \end{subfigure}
     \hfill
     \begin{subfigure}[b]{0.32\textwidth}
         \centering
         \includesvg{figures/results_optiwise_VCT.circle_Y.svg}
        \caption{Sway force.}
        \label{fig:drift_angle_Y_optiwise}
     \end{subfigure}
     \hfill
     \begin{subfigure}[b]{0.32\textwidth}
         \centering
         \includesvg{figures/results_optiwise_VCT.circle_N.svg}
        \caption{Yawing moment.}
        \label{fig:drift_angle_N_optiwise}
     \end{subfigure}
    \caption{Optiwise circle tests from VCT (dots) and predictions (lines).}
    \label{fig:circle_optiwise}
\end{figure}

The coupling terms $Y_{vrr}$,$Y_{vvr}$,$N_{vrr}$, and $N_{vvr}$ in the hull force model (\autoref{eq:Y_H}, \autoref{eq:N_H}) where fitted from the circle and drift variations. These coupling terms are important as shown by the comparison with/without them in \autoref{fig:circle_drift_optiwise}.
%Circle + drift
\begin{figure}[h]
     \centering
     \begin{subfigure}[b]{0.49\textwidth}
         \centering
         \includesvg{figures/results_optiwise_VCT.Y_H.svg}
        \caption{Sway force.}
        \label{fig:circle_drift_Y_H_optiwise}
     \end{subfigure}
     \hfill
     \begin{subfigure}[b]{0.49\textwidth}
         \centering
         \includesvg{figures/results_optiwise_VCT.Y_H_no_coupling.svg}
        \caption{Sway force no coupling.}
        \label{fig:circle_drift_Y_H_no_coupling_optiwise}
     \end{subfigure}

     \vfill
     \begin{subfigure}[b]{0.49\textwidth}
         \centering
         \includesvg{figures/results_optiwise_VCT.N_H.svg}
        \caption{Yawing moment.}
        \label{fig:circle_drift_N_H_optiwise}
     \end{subfigure}
     \hfill
     \begin{subfigure}[b]{0.49\textwidth}
         \centering
         \includesvg{figures/results_optiwise_VCT.N_H_no_coupling.svg}
        \caption{Yawing moment no coupling.}
        \label{fig:circle_drift_N_H_no_coupling_optiwise}
     \end{subfigure}
     
    \caption{Hull forces during the circle and drift variations with/without the coupling terms, VCT (dots), fitted model (surface).}
    \label{fig:circle_drift_optiwise}
\end{figure}

