\noindent After the system identification, various hydrodynamic forces applied on the wPCC could be estimated by either the identified model or VCT analysis. Their comparisons show good agreement as presented in \autoref{fig:results_wPCC_VCT_forces}. However, there are some deviations for the thrust variation tests as shown in Fig.\ref{fig:results_wPCC_VCT_forces}(k)&(l).

\begin{figure}[h!]
    \centering
    \includegraphics[height=5.7in]{figures/results_wPCC_VCT_forces.png}
    \caption{Forces on the wPCC analyzed by VCT (dots) and predictions from the identified model (lines).}
    \label{fig:results_wPCC_VCT_forces}
\end{figure}

The deviations might be caused by the coupling terms $Y_{vrr}$,$Y_{vvr}$,$N_{vrr}$, and $N_{vvr}$ in the hull force model (\autoref{eq:XYN_H}) where fitted from the circle and drift variations. This means that the total hull yawing moment acting on the ship, is only generated by the inviscid Munk moment – which is not included in $N_H$. A nonlinear viscous contribution is present for larger drift angles. The rudders are however the main contributors to the viscous yawing moment. The flow from the port daggerboard hits the starboard rudder for 15-degree drift angle as shown in Fig.\ref{fig:wpcc_drift_flow}.

\begin{figure}[h!]
    \centering
    \includegraphics[width=5in]{figures/paraview_drift_15.png}
    \caption{Flows at the wPCC bottom for VCT at 15 degrees drifting}
    \label{fig:wpcc_drift_flow}
\end{figure}

% \begin{figure}[h]
%      \centering
%      %Drift angle
%      \begin{subfigure}[b]{0.325\textwidth}
%          \centering
%          \includesvg{figures/results_wPCC_VCT.drift_angle_X.svg}
%         \caption{X at various drift angles}
%         \label{fig:drift_angle_X_wPCC}
%      \end{subfigure}
%      \hfill
%      \begin{subfigure}[b]{0.325\textwidth}
%          \centering
%          \includesvg{figures/results_wPCC_VCT.drift_angle_Y.svg}
%         \caption{Y at various drift angles}
%         \label{fig:drift_angle_Y_wPCC}
%      \end{subfigure}
%      \hfill
%      \begin{subfigure}[b]{0.325\textwidth}
%          \centering
%          \includesvg{figures/results_wPCC_VCT.drift_angle_N.svg}
%         \caption{N at various drift angles}
%         \label{fig:drift_angle_N_wPCC}
%      \end{subfigure}
%     % \vfill
    
%      %Circle
%      \begin{subfigure}[b]{0.325\textwidth}
%          \centering
%          \includesvg{figures/results_wPCC_VCT.circle_X.svg}
%         \caption{X at various circle angles}
%         \label{fig:circle_X_wPCC}
%      \end{subfigure}
%      \hfill
%      \begin{subfigure}[b]{0.325\textwidth}
%          \centering
%          \includesvg{figures/results_wPCC_VCT.circle_Y.svg}
%         \caption{Y at various circle angles}
%         \label{fig:circle_Y_wPCC}
%      \end{subfigure}
%      \hfill
%      \begin{subfigure}[b]{0.325\textwidth}
%          \centering
%          \includesvg{figures/results_wPCC_VCT.circle_N.svg}
%         \caption{N at various circle angles}
%         \label{fig:circle_N_wPCC}
%      \end{subfigure}

%     % \vfill
%     %rudder angle
%      \begin{subfigure}[b]{0.325\textwidth}
%          \centering
%          \includesvg{figures/results_wPCC_VCT.rudder_angle_X.svg}
%         \caption{X at various rudder angles}
%         \label{fig:rudder_angle_X_wPCC}
%      \end{subfigure}
%      \hfill
%      \begin{subfigure}[b]{0.325\textwidth}
%          \centering
%          \includesvg{figures/results_wPCC_VCT.rudder_angle_Y.svg}
%         \caption{Y at various rudder angles}
%         \label{fig:rudder_angle_Y_wPCC}
%      \end{subfigure}
%      \hfill
%      \begin{subfigure}[b]{0.325\textwidth}
%          \centering
%          \includesvg{figures/results_wPCC_VCT.rudder_angle_N.svg}
%         \caption{N at various rudder angles}
%         \label{fig:rudder_angle_N_wPCC}
%      \end{subfigure}

%      % \vfill
%      %Thrust variation
%      \begin{subfigure}[b]{0.325\textwidth}
%          \centering
%          \includesvg{figures/results_wPCC_VCT.thrust_variation_X.svg}
%         \caption{X at various thrust}
%         \label{fig:Thrust variation_X_wPCC}
%      \end{subfigure}
%      \hfill
%      \begin{subfigure}[b]{0.325\textwidth}
%          \centering
%          \includesvg{figures/results_wPCC_VCT.thrust_variation_Y.svg}
%         \caption{Y at various thrust}
%         \label{fig:Thrust variation_Y_wPCC}
%      \end{subfigure}
%      \hfill
%      \begin{subfigure}[b]{0.325\textwidth}
%          \centering
%          \includesvg{figures/results_wPCC_VCT.thrust_variation_N.svg}
%         \caption{N at various thrust}
%         \label{fig:Thrust variation_N_wPCC}
%      \end{subfigure}
     
%     \caption{Forces on the wPCC analyzed by VCT (dots) and predictions from the identified model (lines).}
%     \label{fig:VCT_wPCC}
% \end{figure}

%%%%%%%%%%%%%%%%%%%%%%%% Coupling vs uncoupling results %%%%%%%%%%%%%%%%

%Circle + drift
% \begin{figure}[h]
%      \centering
%      \begin{subfigure}[c]{.495\linewidth}
%          \centering
%          \includesvg[width=3.8in, height = 4in]{figures/results_wPCC_VCT.Y_H.svg}
%         \caption{Sway force.}
%         \label{fig:circle_drift_Y_H_wPCC}
%      \end{subfigure}
% %     \hfill
%      \begin{subfigure}[c]{0.495\linewidth}
%          \centering
%          \includesvg[width=3.8in, height = 4in]{figures/results_wPCC_VCT.Y_H_no_coupling.svg}
%         \caption{Sway force no coupling.}
%         \label{fig:circle_drift_Y_H_no_coupling_wPCC}
%      \end{subfigure}

%      \vfill
%      \begin{subfigure}[c]{0.495\linewidth}
%          \centering
%          \includesvg[width=3.8in, height = 4in]{figures/results_wPCC_VCT.N_H.svg}
%         \caption{Yawing moment.}
%         \label{fig:circle_drift_N_H_wPCC}
%      \end{subfigure}
% %     \hfill
%      \begin{subfigure}[c]{0.495\linewidth}
%          \centering
%          \includesvg[width=3.8in, height = 4in]{figures/results_wPCC_VCT.N_H_no_coupling.svg}
%         \caption{Yawing moment no coupling.}
%         \label{fig:circle_drift_N_H_no_coupling_wPCC}
%      \end{subfigure}     
%     \caption{wPCC hull forces during the circle and drift variations with/without the coupling terms, VCT (dots), fitted model (surface).}
%     \label{fig:circle_drift_wPCC}
% \end{figure}

