Similar comparisons are shown for the drift angle tests in \autoref{fig:drift_angle_optiwise} and the circle tests in \autoref{fig:circle_optiwise}. 
\begin{figure}[h]
     \centering
     %Drift angle
     \begin{subfigure}[b]{0.32\textwidth}
         \centering
         \includesvg{figures/results_wPCC_VCT.drift_angle_X.svg}
        \caption{Drift angle X.}
        \label{fig:drift_angle_X_wPCC}
     \end{subfigure}
     \hfill
     \begin{subfigure}[b]{0.32\textwidth}
         \centering
         \includesvg{figures/results_wPCC_VCT.drift_angle_Y.svg}
        \caption{Drift angle Y.}
        \label{fig:drift_angle_Y_wPCC}
     \end{subfigure}
     \hfill
     \begin{subfigure}[b]{0.32\textwidth}
         \centering
         \includesvg{figures/results_wPCC_VCT.drift_angle_N.svg}
        \caption{Drift angle N.}
        \label{fig:drift_angle_N_wPCC}
     \end{subfigure}
    \vfill
    
     %Circle
     \begin{subfigure}[b]{0.32\textwidth}
         \centering
         \includesvg{figures/results_wPCC_VCT.circle_X.svg}
        \caption{Circle X.}
        \label{fig:circle_X_wPCC}
     \end{subfigure}
     \hfill
     \begin{subfigure}[b]{0.32\textwidth}
         \centering
         \includesvg{figures/results_wPCC_VCT.circle_Y.svg}
        \caption{Circle Y.}
        \label{fig:circle_Y_wPCC}
     \end{subfigure}
     \hfill
     \begin{subfigure}[b]{0.32\textwidth}
         \centering
         \includesvg{figures/results_wPCC_VCT.circle_N.svg}
        \caption{Circle N.}
        \label{fig:circle_N_wPCC}
     \end{subfigure}

    \vfill
    %rudder angle
     \begin{subfigure}[b]{0.32\textwidth}
         \centering
         \includesvg{figures/results_wPCC_VCT.rudder_angle_X.svg}
        \caption{Rudder angle X.}
        \label{fig:rudder_angle_X_wPCC}
     \end{subfigure}
     \hfill
     \begin{subfigure}[b]{0.32\textwidth}
         \centering
         \includesvg{figures/results_wPCC_VCT.rudder_angle_Y.svg}
        \caption{Rudder angle Y.}
        \label{fig:rudder_angle_Y_wPCC}
     \end{subfigure}
     \hfill
     \begin{subfigure}[b]{0.32\textwidth}
         \centering
         \includesvg{figures/results_wPCC_VCT.rudder_angle_N.svg}
        \caption{Rudder angle N.}
        \label{fig:rudder_angle_N_wPCC}
     \end{subfigure}

     \vfill
     %Thrust variation
     \begin{subfigure}[b]{0.32\textwidth}
         \centering
         \includesvg{figures/results_wPCC_VCT.thrust_variation_X.svg}
        \caption{Thrust variation X.}
        \label{fig:Thrust variation_X_wPCC}
     \end{subfigure}
     \hfill
     \begin{subfigure}[b]{0.32\textwidth}
         \centering
         \includesvg{figures/results_wPCC_VCT.thrust_variation_Y.svg}
        \caption{Thrust variation Y.}
        \label{fig:Thrust variation_Y_wPCC}
     \end{subfigure}
     \hfill
     \begin{subfigure}[b]{0.32\textwidth}
         \centering
         \includesvg{figures/results_wPCC_VCT.thrust_variation_N.svg}
        \caption{Thrust variation N.}
        \label{fig:Thrust variation_N_wPCC}
     \end{subfigure}

     
    \caption{wPCC tests from VCT (dots) and predictions (lines).}
    \label{fig:VCT_wPCC}
\end{figure}



The coupling terms $Y_{vrr}$,$Y_{vvr}$,$N_{vrr}$, and $N_{vvr}$ in the hull force model (\autoref{eq:Y_H}, \autoref{eq:N_H}) where fitted from the circle and drift variations. These coupling terms are important as shown by the comparison with/without them in \autoref{fig:circle_drift_wPCC}.
%Circle + drift
\begin{figure}[h]
     \centering
     \begin{subfigure}[b]{0.49\textwidth}
         \centering
         \includesvg{figures/results_wPCC_VCT.Y_H.svg}
        \caption{Sway force.}
        \label{fig:circle_drift_Y_H_wPCC}
     \end{subfigure}
     \hfill
     \begin{subfigure}[b]{0.49\textwidth}
         \centering
         \includesvg{figures/results_wPCC_VCT.Y_H_no_coupling.svg}
        \caption{Sway force no coupling.}
        \label{fig:circle_drift_Y_H_no_coupling_wPCC}
     \end{subfigure}

     \vfill
     \begin{subfigure}[b]{0.49\textwidth}
         \centering
         \includesvg{figures/results_wPCC_VCT.N_H.svg}
        \caption{Yawing moment.}
        \label{fig:circle_drift_N_H_wPCC}
     \end{subfigure}
     \hfill
     \begin{subfigure}[b]{0.49\textwidth}
         \centering
         \includesvg{figures/results_wPCC_VCT.N_H_no_coupling.svg}
        \caption{Yawing moment no coupling.}
        \label{fig:circle_drift_N_H_no_coupling_wPCC}
     \end{subfigure}
     
    \caption{wPCC hull forces during the circle and drift variations with/without the coupling terms, VCT (dots), fitted model (surface).}
    \label{fig:circle_drift_wPCC}
\end{figure}

