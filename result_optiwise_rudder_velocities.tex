The rudder inflow angle $\gamma$ and velocity $V_R$ were investigated by conducting a set of separate VCT calculations where the rudder was removed, so that an undisturbed flow by the rudder could be observed. The longitudinal and transverse inflow velocities $u_R$ and $v_R$ were measured along the rudder stock from the root to the tip of the rudder. \autoref{fig:rudder_velocities_span} shows the inflow velocity $\gamma=atan(v_R/u_R)$ along the rudder stock during four of the VCTs. The inflow angle varies a lot along the rudder span, from -27 to +27 degrees, around much smaller average angles of about \textpm 2.5-3 degrees. A positive inflow angle is generated above the clockwise rotating propeller and a negative angle is generated below it.
Rudder forces have been calculated with the MMG rudder model, where the expressions to calculate $u_R$ and $v_R$ have been replaced by measured values. 
\begin{figure}[h]
    \centering 
    \includesvg{figures/results_optiwise_rudder_velocities.inflow_angle.svg}
    \caption{Rudder inflow angle from the root to the tip of the rudder during two circle tests.}
     \label{fig:rudder_velocities_span}
\end{figure}
\begin{figure}[h]
     \centering
     \begin{subfigure}[b]{\textwidth}
         \centering
         \includesvg{figures/results_optiwise_rudder_velocities.drift_angle.svg}
        \caption{Drift angle.}
        \label{fig:inflow_to_force_drift_angle}
     \end{subfigure}
     \vfill
     \begin{subfigure}[b]{\textwidth}
         \includesvg{figures/results_optiwise_rudder_velocities.circle.svg}
        \caption{Circles.}
        \label{fig:inflow_to_force_circle}
     \end{subfigure}
        \caption{Rudder forces from VCT and predictions from the measured inflow velocities.}
        \label{fig:inflow_to_rudder_force}
\end{figure}
