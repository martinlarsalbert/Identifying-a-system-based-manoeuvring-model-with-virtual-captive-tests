\noindent Two test models/cases have been studied in this paper, named as wPCC and Optiwise. The wPCC test case is the ship model that was designed for wind-assisted propulsion systems (WAPS) and can alter between a fully sailing mode, and a fully motoring mode, and in between. 
However, this paper only considers the motoring mode. The wPCC design differs from conventional motoring cargo ship designs in that the ship has two very large rudders, they are actually two to three times larger than needed for a conventional ship. The ship also has fins at the bilge to generate extra lift while sailing, as shown on the scale model in \autoref{fig:wPCC}.
\autoref{tab:main_particulars} shows the main particulars of the scale model. 

\begin{figure}[h]
    \centering
    \includegraphics[width=\columnwidth]{figures/5m2.jpg}
    \caption{Scale model of the wPCC used in the model tests. Copyright RISE.}
    \label{fig:wPCC}
\end{figure}

The Optiwse test case is an ordinary VLCC tanker but with a larger rudder size adopted for WAPS as shown in the scale model in \autoref{fig:optiwise} with main particulars according to \autoref{tab:main_particulars}.
FRMTs with Optiwise were conducted as a continuation of the wPCC experiments, with a more conventional ship design. The experiments were run at a lower Froude number, compared to wPCC, so that wave generation and roll would have smaller impacts. The larger rudder was expected to play a more important part of the total hydrodynamics, compared to conventional ships. The model was therefore equipped with rudder force transducers, so that the rudder forces could be observed during the maneuvers. 
\begin{figure}[h]
    \centering
    \includegraphics[width=\columnwidth]{figures/optiwise.jpg}
    \caption{Scale model of the Optiwise used in the model tests. Copyright RISE.}
    \label{fig:optiwise}
\end{figure}
\begin{table}[h]
    \centering
    \caption{Main particulars (SI units) of the wPCC scale model.}
    \label{tab:main_particulars}
    \pgfplotstabletypeset[col sep=comma, column type=r,
        columns/Parameter/.style={column type=l,string type},
        columns/Unit/.style={column type=l,string type,column name=~},
        columns/Description/.style={column type=l,string type},
        columns/Value/.style={column type=r, column name=~},
        every head row/.style={before row=\hline,after row=\hline},
        every last row/.style={after row=\hline}
    ]{tables/test_cases.main_particulars.csv}
\end{table}